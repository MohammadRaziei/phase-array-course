%----------------------------------------------------------------------------------------
%	PACKAGES AND OTHER DOCUMENT CONFIGURATIONS
%----------------------------------------------------------------------------------------

\documentclass[paper=a4, fontsize=11pt]{scrartcl} % A4 paper and 11pt font size
\usepackage{graphicx}
\usepackage[T1]{fontenc} % Use 8-bit encoding that has 256 glyphs
\usepackage{fourier} % Use the Adobe Utopia font for the document - comment this line to return to the LaTeX default
\usepackage[english]{babel} % English language/hyphenation
\usepackage{amsmath,amsfonts,amsthm} % Math packages

\usepackage{lipsum} % Used for inserting dummy 'Lorem ipsum' text into the template
\usepackage{listings}
\usepackage{sectsty} % Allows customizing section commands
\allsectionsfont{\centering \normalfont\scshape} % Make all sections centered, the default font and small caps

\usepackage{fancyhdr} % Custom headers and footers
\pagestyle{fancyplain} % Makes all pages in the document conform to the custom headers and footers
\fancyhead{} % No page header - if you want one, create it in the same way as the footers below
\fancyfoot[L]{} % Empty left footer
\fancyfoot[C]{} % Empty center footer
\fancyfoot[R]{\thepage} % Page numbering for right footer
\renewcommand{\headrulewidth}{0pt} % Remove header underlines
\renewcommand{\footrulewidth}{0pt} % Remove footer underlines
\setlength{\headheight}{13.6pt} % Customize the height of the header

\numberwithin{equation}{section} % Number equations within sections (i.e. 1.1, 1.2, 2.1, 2.2 instead of 1, 2, 3, 4)
\numberwithin{figure}{section} % Number figures within sections (i.e. 1.1, 1.2, 2.1, 2.2 instead of 1, 2, 3, 4)
\numberwithin{table}{section} % Number tables within sections (i.e. 1.1, 1.2, 2.1, 2.2 instead of 1, 2, 3, 4)

\setlength\parindent{0pt} % Removes all indentation from paragraphs - comment this line for an assignment with lots of text

%----------------------------------------------------------------------------------------
%	TITLE SECTION
%----------------------------------------------------------------------------------------

\newcommand{\horrule}[1]{\rule{\linewidth}{#1}} % Create horizontal rule command with 1 argument of height

\title{	
\normalfont \normalsize 
\textsc{Sharif University of Technology\\ Electrical Engineering Dept.} \\ [25pt] % Your university, school and/or department name(s)
\horrule{0.5pt} \\[0.4cm] % Thin top horizontal rule
\huge Phased Array Systems\\ Assignment 2, System Design  \\ % The assignment title
\horrule{2pt} \\[0.5cm] % Thick bottom horizontal rule
}
%========= TYPE YOUR NAME HERE ==================
\author{Ali Hosseini} % Your name

\date{\normalsize\today} % Today's date or a custom date

\begin{document}


\maketitle % Print the title
\begin{figure}[!t]
\centering
\includegraphics[scale=0.65]{Sharif}
\end{figure}
%---------------------------------------------------------------------------------------
%	PROBLEM 1
%----------------------------------------------------------------------------------------
\section{Friis Link Equation and Phased Array Design}

A small-cell backhaul company wants to design a 60 GHz link between two stations which are 1.2 km apart. The objective is to transmit and receive minimum 1 Gbps data. \\

\subsection{Single Antenna Implementation}\label{SISO}
a)If CMOS technology is used, what is the minimum antenna gain for a Single antenna at TX and RX (SISO)? \\
b) What is the receiver sensitivity?\\

Hints:\\
\begin{enumerate}
\item CMOS power amplifiers give an output power ($P_{1dB}$) of typically 0 to 8 dBm at 60 GHz.
\item Assume QPSK modulation, and find minimum required SNR for BER of $10^{-5}$.
\item Calculate the physical bandwidth assuming 25\% coding overhead.
\item Find Oxygen ($O_2$) absorption loss at 60 GHz band.
\item Assume clear sky condition (no rain).
\item Assume perfect antenna alignment and matching. 
\item  Plot antenna gain as a function of $P_{1dB}$ and receiver noise figure ($F$ ranges from 5 to 10 dB).
\item You are allowed to make any reasonable assumption if necessary. But you should clearly justify your assumption.
\end{enumerate}


\subsection{SISO:  Pole Sway and antenna misalignment}
Fig. \ref{Fig4} shows pole sway caused by wind. Measurements show that pole sway can be as large as $\pm2.7^\circ$. Assume Parabolic antennas are used in Section \ref{SISO}. For a Parabolic antenna 3-dB beamwidth in degree is approximately given by:

\begin{figure}[h]
\centering
\includegraphics[scale=0.8]{lighting-poles-sway.png}
\caption{Pole sway illustration.}
\label{Fig4}
\end{figure}


\begin{equation}
\Delta\theta=\frac{70\lambda}{L}
\end{equation}
where $\Delta\theta$ denotes antenna beamwidth and $L$ is the antenna diameter.\\
You can calculate the gain of a Parabolic antenna from here:\\ $http://www.qsl.net/pa2ohh/jsparabolic.htm$
\\ or use the following relation: 

 \begin{equation}
 G=\eta\times\frac{\pi^2L^2}{\lambda^2}
 \end{equation}
 where $\eta$, known as the aperture efficiency, is typically 0.55 to 0.70.


How much drop in the received SNR is caused by maximum antenna pole sway? What solutions do you recommend (at least 3 solutions)?\\

Hints: \\

You should approximate the antenna gain with a linear function of angle.

\subsection{Phased Array Solution: Clear Sky}\label{clear}
 
Design a CMOS TX phased array and a CMOS RX phased array at 60 GHz for 1 Gbps data at 1 km. Calculate gain drop for maximum sway angle with and without beamforming. Explain your design steps and assumptions. 

\section{Qualcomm 28 GHz Phased Array for 5G (Midterm 2016)}

In Nov 2015, Qualcomm company demonstrated their TDD synchronous system operating in the 28 GHz band. The demonstration included one millimeter-wave (mm-wave) base station and one mobile device. The mm-wave base station, had 128 antenna elements with 16 controllable RF channels, while the device contained four sub-arrays with 4 controllable RF channels.\\

1-Search to find more information about this project. Explain why 28 GHz frequency was chosen?\\

2- Why only 4 sub-arrays were used for the mobile device? Does increasing the carrier frequency help to have a better link?\\

3- In a LOS link measurement maximum range of 350 m was measured. Find the total transmitted power assuming 500 MHz bandwidth, 10 dB noise figure, 5 dBi antenna element gain, and minimum  10 dB SNR at the receiver output.\\

4- What is the maximum LOS range between two mobile devices?\\

5- One requirement for the base station is to generate multiple simultaneous beams. What is your hardware architecture solution to achieve this goal?\\

\section{Samsung 28 GHz Phased Array for 5G (Midterm 2016)}

For several years, Samsung has been developing 28 GHz phased array solutions for mobile communications. In 2014, Samsung demonstrated a phased array link for the future 5G mobile communication at 28 GHz frequency. Both transmit and receive array antennas had the same number of antennas arranged in the form of a uniform planar array, confined within an area of $60 mm
\times 30 mm$. The total array gain was 18 dBi.\\

1- Plot the 3D radiation pattern and calculate maximum gain of a uniform planar array with 25 element ($5\times 5$) using patch elements with $\lambda/2$ spacing. \\

2- If 200 m LOS range is required, what is the number of antennas, element gain and TX Effective isotropic radiated power (EIRP)? This problem might have many solutions. You should justify your answer.\\

3- In order to reduce the hardware complexity, a sub-array architecture was employed to group antennas into a sub-array. What are the side effects of using sub-arrays instead of independent elements? Name three possible issues.\\

4- Assume each 2 by 1 power combiner/splitter has $1.2 d$B loss. What is your proposed optimum size for a sub-array?\\

\section{Dolph-Chebychev Tapering}
Study \textbf{Section 23.9} of the enclosed file entitles as \textit{Array Design Methods} to learn Dolph-Chebychev array weighting.\\

Consider a standard 11-element linear array pointed at broadside.\\

4-a) Calculate the Dolph-Chebychev weightings for sidelobe levels of -20 dB, -30 dB, and -40 dB. Write them in a Table.\\

4-b) Plot the resulting beam pattern and compute the HPBW, BWNN, and the peal array gain for each weighting.\\


You should develop your own code in Matlab to calculate array weights. Zip your codes with your homework solution. 

\section{Frequency Scanning}

Fig. \ref{Fig1} shows a uniform linear array with spacing d.

\begin{figure}[h]
\centering
\includegraphics[scale=0.35]{HW2_1.jpg}
\caption{Uniform linear array}
\label{Fig1}
\end{figure}

\subsection{Array Factor}
For $f_0= 60$ GHz, $N=10$, and $d=\lambda_0/2$ plot array factor in $\theta$ plane (i.e. $\phi=0$) when:\\ 
a) $\Delta\phi$=$\pi/2$\\
b) $\Delta\phi$=$\pi$\\
where $\Delta\phi$ is the progressive phase shift. For each case determine the steering angle, sidelobe level, and HPBW. Assume all antennas are isotropic.\\

(c)For the same array, fix $\Delta\phi$ at $\pi/4$ and change the frequency from 50 to 70 GHz.\\  Plot steering angle (peak gain direction) versus frequency.\\

d) Now, consider the antenna shown in Fig. \ref{Fig3}. Assume antenna spacing is $\lambda_0/2$, and the inter-element feed section is  $\lambda_0$, where $\lambda_0$ is the wavelength at 60 GHz. Plot the steering angle when the frequency changes from 50 to 70 GHz. Compare your results with part (c) and discuss the differences.

\begin{figure}[h]
\centering
\includegraphics[scale=0.5]{Sfed}
\caption{Series-Fed array}
\label{Fig3}
\end{figure}



%----------------------------------------------------------------------------------------

\end{document}