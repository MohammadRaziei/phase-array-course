\documentclass{utsignal}

\usepackage{amsmath}
\usepackage{hyperref}
\usepackage{wrapfig}
\usepackage{verbatim}
\usepackage{fancyvrb}
\usepackage{lscape}
\usepackage{rotating}
\usepackage{xepersian}
\usepackage{listings}
\usepackage{color}
\usepackage[utf8]{inputenc}

\title{تمرین شماره‌ی ۴}
\course{سیستم‌های آرایه فازی}
\lecturer{دکتر محمد فخارزاده}
\deadline{سه‌شنبه ۲۵ دی ۱۴۰۳، ساعت ۲۳:۵۵}
\begin{document}
    \maketitle
    \section{بیان مسئله}
    همان‌طور که در درس توضیح داده شده است، تلف عبوری فازگران‌های
    \lr{RTPS}
   به ازای ولتاژهای کنترلی مختلف تغییراتی غیرخطی و غیریکنوا دارد. همین موضوع سبب می‌گردد تا پرتوسازی همدوس لزوما به نتیجه مطلوب منجر نگردد. بنابراین پرتوسازی در سامانه‌های آرایه فازی به یک مسئله بهینه‌سازی غیرخطی و غیرمحدب با اسکترمم‌های محلی زیاد تبدیل می‌شود که یافتن پاسخ بهینه آن به سادگی ممکن نیست. در این تمرین قرار است با تعدادی از روش‌های پرتوسازی کلاسیک آشنا شده و در نهایت نتیجه آن را با مدل یادگیری عمیق مقایسه کنیم. \\
   ساختار پیشنهادی، مطابق شکل
   \ref{fig:system}
   یک سامانه آرایه فازی با فازگردان‌های
   \lr{RTPS}
   واقعی است که مشخصات دامنه و فاز آن در اختیار دانشجویان قرار داده می‌شود. این سامانه از ۴ آنتن همسانگرد که روی محور
   \lr{z}
   و به مرکز مبداء مختصات قرار گرفته‌اند، تشکیل شده است. آنتن‌ها به صورت یکنواخت با فاصله نصف طول موج از یکدیگر قرار دارند. سیگنال‌های خروجی آنتن‌ها پس از تغییر فاز با یکدیگر جمع‌ شده و سیگنال
   $S$
   را می‌سازند. منبع مورد نظر در زاویه
   $\theta$
   نسبت به راستای آرایه قرار گرفته و موج صفحه‌ای آن بر آرایه می‌تابد. از طریق تنظیم ولتاژهای کنترلی با ۴ بیت در بازه ۰ تا ۱۰ ولت قرار است یکی از دو هدف زیر (بر اساس شماره دانشجویی) دنبال شود:
    \begin{enumerate}
    	\item دامنه سیگنال خروجی 
    	$S$
    	به ازای هر زاویه
    	$\theta$
    	بیشینه شود. (شماره دانشجویی‌ فرد)
    	\item یک
    	\lr{null}
    	در جهت ورود سیگنال قرار گرفته و بنابراین دامنه سیگنال خروجی کمینه شود. (شماره دانشجویی ‌زوج)
    \end{enumerate}
   در هدف اول قرار است تا با بیشینه کردن بهره آرایه در جهت ورود سیگنال،
   \lr{SNR}
   نیز در خروجی بیشینه گردد. در هدف دوم سیگنال ورودی به عنوان سیگنال اختلال در نظر گرفته شده است که سعی می‌شود تا جای ممکن دامنه آن کاهش یابد. \\
    به این منظور از سیگنال‌های دریافتی نمونه‌برداری شده و به باند پایه منتقل می‌گردد. به این ترتیب سیگنال‌های ورودی نمونه‌برداری شده و به صورت زیر قابل بیان است. توجه شود که انتقال سیگنال به باند پایه مستلزم مختلط شدن آن است.
    \begin{equation}
    	s_i = e^{-j\phi_i} + n_i, \quad 0\leq i \leq N - 1
    \end{equation}
    در رابطه فوق
    $n_i$
    نویز سفید گاوسی است. در روش‌های کلاسیک پرتوسازی مانند پرتوسازی همدوس و 
    \lr{zero-knowledge}
    ولتاژهای کنترلی فازگردان‌ها به نحوی تغییر می‌یابد تا دامنه سیگنال
    $S$
    بهینه گردد. این در حالی است که در روش یادگیری عمیق با نظارت، بخش‌های حقیقی و موهومی سیگنال‌های
    $s_i$
    به عنوان ورودی شبکه و ولتاژهای کنترلی به عنوان خروجی‌های آن خواهد بود. تجزیه سیگنال‌های
    $s_i$
    به بخش‌های حقیقی و موهومی به صورت زیر انجام می‌شود:
    \begin{equation}
    	Re(s_i) = cos(\phi_i) + n_{iR}, \quad Im(s_i) = -sin(\phi_i) + n_{iI}
    \end{equation}
    بنابراین در هر لحظه یک بردار ۸ تایی از مقادیر حقیقی حاصل می‌شود.
    \begin{figure}[H]
    	\centering
    	\includegraphics[width=0.6\linewidth]{system.png}
    	\caption{ساختار آرایه فازی و مدل یادگیری عمیق}
    	\label{fig:system}
    \end{figure}
    \section{الگوریتم‌های پرتوسازی کلاسیک}
    \begin{enumerate}
    	\item با استفاده از فایل‌های
    	\lr{AmpW.mat}
    	و
    	\lr{PhasW.mat}
    	نمودار تغییرات دامنه و فاز را برحسب ولتاژهای ۰ تا ۱۰ ولت رسم کنید. در دو فایل بیان شده مشخصات ۱۷ فازگردان مشابه در ۲۰۱ نقطه فرکانسی به ازای ولتاژهای کنترلی ۰ تا ۱۰ ولت قرار داده شده است. از میان اندازه‌گیری‌های صورت گرفته، چهار فازگردان را به دلخواه انتخاب کنید تا در آرایه فازی مورد استفاده قرار گیرد و مشخصات یکی از آن‌ها را نقطه فرکانسی وسط رسم کنید.
    	\item از روش پرتوسازی همدوس استفاده کنید و ولتاژهای بهینه و بهره آرایه را به صورت تئوری و شبیه‌سازی محاسبه کنید.
    	\item روش پرتوسازی
    	\lr{zero-knowledge}
    	را برای دو هر دو حالت بهینه‌سازی گرادیان نزولی
    	\footnote{\lr{gradient descent}}
    	\lr{one-sided}
    	و
    	\lr{two-sided}
    	و به ازای سه مقدار متفاوت برای گام بهینه‌سازی که در درس با
    	$\mu$
    	نمایش داده شده است، پیاده‌سازی کنید. توجه کنید که ولتاژهای کنترلی به صورت گسسته و با چهار بیت تعیین می‌شود.
    	\item جهت ورود سیگنال را از
    	$\theta = 0^{\circ}$
    	تا
    	$\theta = 180^{\circ}$
    	با گام‌های
    	$0.1^{\circ}$
    	تغییر دهید و به ازای هر زاویه ولتاژهای بهینه متناظر با هر فازگردان را از بین همه حالت‌های ممکن انتخاب کنید. به این روش بهینه‌سازی که با هزینه محاسباتی بسیار بالایی همراه است، جستجوی جامع
    	\footnote{\lr{exhaustive search}}
    	گفته می‌شود. با توجه به زمان‌بر بودن این بخش پیشنهاد می‌شود نتیجه را ذخیره کنید تا در بخش‌های دیگر با فراخوانی مورد استفاده قرار گیرد.
    	\item بهره خروجی سه روش بیان شده را برای حالت‌های مختلف و به ازای جهت‌های ورودی
    	$30^{\circ}$،
    	$60^{\circ}$
    	و
    	$90^{\circ}$
    	رسم کنید. در این نمودارها محور افقی زاویه و محور عمودی نمایش لگاریتمی بهره آرایه است. نتایج را با یکدیگر مقایسه کنید و اثر مقادیر مختلف
    	$\mu$
    را در بهره و تعداد گام‌های همگرایی توجیه کنید.
    \end{enumerate}
    \section{الگوریتم پرتوسازی مبتنی بر یادگیری عمیق}
    همان‌طور که پیش‌تر بیان شد، هدف اصلی این تمرین مقایسه نتایج الگوریتم‌های کلاسیک و یادگیری عمیق است. به این منظور گام‌های زیر برای تولید داده و طراحی مدل یادگیری عمیق با نظارت پیشنهاد شده است.
    \begin{enumerate}
    	\item ورودی مدل یادگیری عمیق، برداری ۸ تایی از سیگنال‌های دریافتی روی ۴ آنتن آرایه فازی است. برای تولید داده آموزش و تست، جهت ورود سیگنال را در بازه
    	$\theta = 0^{\circ}$
    	تا
    	$\theta = 180^{\circ}$
    	با گام‌های
    	$0.1^{\circ}$
    	تغییر دهید و بردار ورودی مدل را به ازای هر حالت ذخیره کنید. به منظور بررسی اثر نویز بر عملکر مدل و همچنین افزایش تعداد داده، به ازای هر زاویه ورود،
    	\lr{SNR}
    	سیگنال را در بازه
    	$5dB$
    	تا
    	$20dB$
    	با گام‌های
    	$0.5dB$
    	تغییر دهید و سیگنال‌های حقیقی و موهومی ورودی مدل را شبیه‌سازی کنید.
    	\item با توجه به اینکه در این تمرین قرار است تا از مدل یادگیری عمیق با نظارت استفاده شود، لازم است تا داده شبیه‌سازی شده در بخش قبل برچسب‌گذاری شود. به این منظور از الگوریتم جستجوی جامع که در بخش اول تمرین توسعه دادید استفاده کنید تا ولتاژهای بهینه مربوط به هر زاویه ورود را به داده متناظر با آن اضافه کنید. واضح است که ولتاژهای بهینه تنها به ازای زوایای ورود سیگنال تغییر می‌کند و تغییر
    	\lr{SNR}
    	تاثیری بر روی آن ندارد.
    	\item از آنجا که قالب ولتاژ در خروجی شبکه عصبی تاثیر به‌سزایی در عملکرد شبکه دارد، پیشنهاد می‌شود از 
    	\lr{Gray Code}
    	برای ولتاژها استفاده کنید. روش تبدیل بیت‌های معمولی به
    	\lr{Gray Code}
    	را بیان کرده و مزیت آن را شرح دهید.
    	\item مدل یادگیری عمیقی طراحی کنید که در آن از روی بردار ۸ تایی ورودی، بیت‌های مربوط به ولتاژهای کننرلی خروجی تعیین گردد. به این منظور پیشنهاد می‌شود از ساختار اتوانکودر - دیکودر به دلیل مقاوم بودن آن در برابر نویز استفاده کنید. البته مانعی برای استفاده از دیگر معماری‌های یادگیری وجود ندارد. ساختار لایه‌های مدل طراحی شده را با بهره‌گیری از ابزارهای گرافیکی رسم کنید.
    \end{enumerate}
    \section{مقایسه نتایج}
    \begin{enumerate}
    	\item با تقسیم داده شبیه‌سازی شده در بخش قبل، ابتدا شبکه طراحی شده را آموزش داده و سپس آن را تست کنید. براساس شباهت ولتاژهای خروجی داده تست با برچسب‌های متناظر، خطای
    	\lr{RMS}
    	شکبه عصبی را محاسبه کنید. با تغییر لایه‌ها، تعداد نورون‌ها و توابع فعال‌ساز تلاش کنید تا به دقت بیشینه دست یابید و نتایج بهبود یافته را گزارش کنید.
    	\item بهره بدست آمده از خروجی شبکه عصبی را در زوایای ورودی
    	$30^{\circ}$،
    	$60^{\circ}$
    	و
    	$90^{\circ}$
    	با الگوریتم‌های پرتوسازی کلاسیک مقایسه کنید.
    	\item با ایجاد شرایطی یکسان برای اجرای هر یک از روش‌های کلاسیک و یادگیری عمیق بیان شده و همچنین با گزارش مدل
    	\lr{CPU}
    	و حجم
    	\lr{RAM}
    	خود، روش‌های بیان شده را از جهت زمان اجرا مقایسه کنید.
    \end{enumerate}
    \section{نگارش گزارش و توضیحات}
    در انجام تمرین و نگارش گزارش به موارد زیر دقت کنید:
   	\begin{enumerate}
   		\item برای پیاده‌سازی هر یک از بخش‌های بیان شده می‌توانید از
   		\lr{MATLAB}
   		یا
   		\lr{Jupyter Notebook}
   		استفاده کنید. در صورت استفاده از
   		\lr{Jupyter Notebook}
   		دستورات مربوط به نصب پکیج‌های موردنیاز را در بخش اول کد بیاورید تا امکان اجرای کد بدون نیاز به هیچ دستور اضافه‌ای وجود داشته باشد. سعی کنید تا کد را با بهره‌گیری از کامنت‌های مناسب تا جای ممکن خوانا کنید.
   		\item فایل فشرده نهایی و تمامی فایل‌های خود را به صورت انگلیسی نام‌گذاری کنید. 
   		\item  فایل‌‌های خود را در پوشه‌ای که با شماره دانشجویی شما نامگذاری شده‌ است قرار دهید و در قالب
   		\lr{zip}
   		در سامانه درس بارگذاری کنید. گزارش‌های خود را در قالب
   		\lr{LaTex}
   		بنویسید و علاوه بر خروجی
   		\lr{PDF}،
   		فایل‌های اصلی را نیز قرار دهید.
   		\item این تمرین
   		\boldmath{تحویل حضوری ندارد}. بنابراین سعی‌ کنید گزارش خود را کامل بنویسید.
   		\item سوالات خود را از طریق ایمیل روبرو بپرسید:
   		\href{mailto:a.aghaei.s@gmail.com}{\lr{a.aghaei.s@gmail.com}}
   	\end{enumerate}
  موفق و سلامت باشید
  \end{document}
