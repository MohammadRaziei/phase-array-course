%----------------------------------------------------------------------------------------
%	PACKAGES AND OTHER DOCUMENT CONFIGURATIONS
%----------------------------------------------------------------------------------------

\documentclass[paper=a4, fontsize=11pt]{scrartcl} % A4 paper and 11pt font size
\usepackage{graphicx}
\usepackage[T1]{fontenc} % Use 8-bit encoding that has 256 glyphs
\usepackage{fourier} % Use the Adobe Utopia font for the document - comment this line to return to the LaTeX default
\usepackage[english]{babel} % English language/hyphenation
\usepackage{amsmath,amsfonts,amsthm} % Math packages

\usepackage{lipsum} % Used for inserting dummy 'Lorem ipsum' text into the template
\usepackage{listings}
\usepackage{sectsty} % Allows customizing section commands
\allsectionsfont{\centering \normalfont\scshape} % Make all sections centered, the default font and small caps

\usepackage{fancyhdr} % Custom headers and footers
\pagestyle{fancyplain} % Makes all pages in the document conform to the custom headers and footers
\fancyhead{} % No page header - if you want one, create it in the same way as the footers below
\fancyfoot[L]{} % Empty left footer
\fancyfoot[C]{} % Empty center footer
\fancyfoot[R]{\thepage} % Page numbering for right footer
\renewcommand{\headrulewidth}{0pt} % Remove header underlines
\renewcommand{\footrulewidth}{0pt} % Remove footer underlines
\setlength{\headheight}{13.6pt} % Customize the height of the header

\numberwithin{equation}{section} % Number equations within sections (i.e. 1.1, 1.2, 2.1, 2.2 instead of 1, 2, 3, 4)
\numberwithin{figure}{section} % Number figures within sections (i.e. 1.1, 1.2, 2.1, 2.2 instead of 1, 2, 3, 4)
\numberwithin{table}{section} % Number tables within sections (i.e. 1.1, 1.2, 2.1, 2.2 instead of 1, 2, 3, 4)

\setlength\parindent{0pt} % Removes all indentation from paragraphs - comment this line for an assignment with lots of text

%----------------------------------------------------------------------------------------
%	TITLE SECTION
%----------------------------------------------------------------------------------------

\newcommand{\horrule}[1]{\rule{\linewidth}{#1}} % Create horizontal rule command with 1 argument of height

\title{	
\normalfont \normalsize 
\textsc{\textbf{Sharif University of Technology}\\ Electrical Engineering Department \\Phased Array Systems, Fall 2024} \\ [35pt] % Your university, school and/or department name(s)
\horrule{0.5pt} \\[0.4cm] % Thin top horizontal rule
\huge Assignment 1  \\ % The assignment title
\horrule{2pt} \\[0.5cm] % Thick bottom horizontal rule
}
%========= TYPE YOUR NAME HERE ==================
\author{John Smith} % Your name

\date{\normalsize\today} % Today's date or a custom date

\begin{document}

\maketitle % Print the title

%---------------------------------------------------------------------------------------
%	PROBLEM 1
%----------------------------------------------------------------------------------------
\section{Perfect Array}
Consider a non-uniform 4-element linear array, whose element separations are $d$, $3d$, and $2d$ where $d = \lambda/2$, as depicted in Fig. \ref{perfect}. The antenna elements are weighted uniformly.\\

\begin{figure}
\centering
\includegraphics[scale=0.6]{Perfect}
\caption{4-element perfect array.}
\label{perfect}
\end{figure}
(a) Compute the beam pattern and the null to null beamwidth, $BW_{NN}$ of this 4-element array.\\

(b) Compare the results in (a) with a uniform 7-element uniform array with $d = \lambda/2$  spacing. Discuss the behavior of the main lobe and the sidelobes of both arrays.\\

c) Is beamwidth a function of number of antennas or array length?

%---------------------------------------------------------------------------------------
%	PROBLEM 2
%----------------------------------------------------------------------------------------
\section{Antenna Pattern Calculation}
Download $Trogon.csv$ file form HW1 package. This file contains the 3D pattern data of a parasitic patch antenna generated by HFSS at 60.48 GHz, illustrated in Fig. \ref{Fig2}\\

\begin{figure}[!h]
\centering
\includegraphics{Trogon_Antenna.png}
\caption{Two parasitic patch antennas in a $7\times 7 mm^2$ organic package used for TX and RX operation. }
\label{Fig2}
\end{figure}

(a)- what is the size (angular range) and angular resolution of this antenna radiation pattern?\\ 

(b)- Change data format to a square matrix such that one dimension shows $\phi$ angle and the other one $\theta$ angle.\\ 
Hint: Check file header first. You can use \emph{reshape}  function in Matlab.\\

Plot 3D pattern using \emph{IMAGESC} command. Label axis accordingly and show colorbar.\\

(c)- Plot the 2D antenna pattern for $\phi =0^\circ$. Calculate the antenna peak gain, HPBW, null level and sidelobe level in this plane. Ripples in HPBW of the antenna are not sidelobes\\

(d)- Plot the 2D antenna pattern for $\phi =90^\circ$. Calculate the antenna peak gain, HPBW, null level and sidelobe level in this plane.\\

(e)- How do you think antenna mutual coupling has affected the antenna pattern.\\

(f) Construct a 4$\times$4 square array with this element. Using the principle of pattern multiplication, for uniform weighting, calculate the peak gain, and HPBW of the array. Assume element spacings in x and y directions are  $d_x=7 mm$ ,  and $d_y=5 mm$.

%----------------------------------------------------
%	PROBLEM 3
%----------------------------------------------------------------------------------------

\section{Antenna Bandwidth Calculation}

Download $Trogon_Sparam.csv$ file. This file contains the simulated \emph{Scattering matrix} of two  parasitic patch antennas, which are 1.9 mm apart, generated by HFSS over 60 GHz band. Fig. \ref{Fig2} shows the top view of the package with two antennas.\\


(a)- what is the frequency range of S-parameters?\\

(b)- Plot S-parameters and guess what each column of the S-parameter matrix represents. Which curves are similar? why?\\

(c)- Calculate the 10-dB impedance bandwidth of each antenna. What is the fractional bandwidth of this antenna?\\

(d)- Calculate the antenna coupling at the resonant frequency. How can we decrease the antenna coupling?
%----------------------------------------------------
%	PROBLEM 4
%----------------------------------------------------------------------------------------

\section{Array Beam Generation} 
Download $Array\_beam\_cal$ Matlab function from CW.
This function has 4 inputs:
 \begin{enumerate} 
  \item $R$ which is a $N\times3$ matrix. Each row gives $x$, $y$ and $z$ coordinates of one antenna
  \item $phases$ which is a $1\times N$ vector. Each element is the phase applied to one antenna
  \item $Theta$ is the $\theta$ coordinate of the observation point.
   \item $Phi$ is the $\phi$ coordinate of the observation point in Spherical coordinates system.
\end{enumerate}
a) Calculate array factor for $\phi=0^\circ$ and $-180 ^\circ \leq \theta \leq 180^\circ$, assuming zero phases are applied to antennas located with $R1$
\begin{lstlisting}
 R1=(2.5e-3)*
 [3 0 0
  2 0 0
  1 0 0
  0 0 0
  -1 0 0
  -2 0 0
  -3 0 0]
\end{lstlisting}
b) Repeat part a for $R2$ given below and compare the results.
\begin{lstlisting}
 R2=(5e-3)*
 [3 0 0
  2 0 0
  1 0 0
  0 0 0
  -1 0 0
  -2 0 0
  -3 0 0]
\end{lstlisting}

c) Repeat part a for $R3$ given below and compare the results with part b.
\begin{lstlisting}
R3=(5e-3)*
[3+0.5*randn() 0 0
 2+0.5*randn() 0 0
 1+0.5*randn() 0 0
 0+0.5*randn() 0 0
 -1+0.5*+randn() 0 0
 -2+0.5*randn() 0 0
 -3+0.5*randn() 0 0]
    \end{lstlisting}
d) Repeat part c for many times ($ > 100$) and select the best results. Explain what you learned from this problem.
%----------------------------------------------------
%	PROBLEM 5
%----------------------------------------------------------------------------------------    
    \section{Synthetic-aperture radar (SAR)} 
%------------------------------------------------
Synthetic-aperture radar (SAR) is a form of radar that is used to create two or three-dimensional images of objects, such as landscapes. SAR images have wide applications in remote sensing and mapping of the surfaces of both the Earth and other planets. Some of the other important applications of SAR are topography, oceanography, glaciology, geology (for example, terrain discrimination and subsurface imaging), and forestry, which includes forest height, biomass, deforestation. Volcano and earthquake monitoring is a part of differential interferometry. It is also useful in environment monitoring like oil spills, flooding, urban growth, global change and military surveillance, which includes strategic policy and tactical assessment.\\


a) A SAR system is installed on a satellite at 600 km height above ground level and desired resolution of generated image is 1 m. Calculate maximum beam width of scanning antenna.\\

b) If a ULA is used in a), calculate the minimum number of antennas and array length. (operating frequency is 12 GHz).\\

c) Compare the calculated array length with a satellite dimensions, and explain how to create very small beam width in SAR, with the help of phased arrays.\\

d) PAMIR is a phased array SAR. Find PAMIR charactereistics such as beamwidth, frequency, bandwidth, 

%----------------------------------------------------
%	PROBLEM 6
%----------------------------------------------------------------------------------------    
    \section{Array Geometry} 
%------------------------------------------------
Array geometry means how the antenna elemnts of an array are arranged. It plays a key role in determining the array beam. In this problem we investigate and compare the array patterns of three famous geometries\\


a) \textbf{Uniform Linear Array}: Plot the array factor of a 25 element, with d=$\lambda/2$. Calculate the SLL, HPBW and BWNN. \\

b) \textbf{Uniform Circular Array}: In this case array elements are placed on a circle with identical spacing. Calculate the radius of a 25 element UCA whose element spacing is d=$\lambda/2$.
Plot the array factor . Calculate the SLL, HPBW and BWNN.\\

c) \textbf{Square Array}: In this case array elements form a square. Plot the array factor of a $5\times 5$ element square array, with d=$\lambda/2$. Calculate the SLL, HPBW and BWNN.\\

d) Compare  SLL, HPBW and BWNN values of these three structures in a Table.


%----------------------------------------------------------------------------------------

\end{document}