\documentclass[12pt,onecolumn,a4paper]{article}
\usepackage{amsthm,amsmath,amssymb,bm}
\usepackage{epsfig,graphicx,subcaption}
\usepackage{float}
\usepackage{color,xcolor}
\usepackage{fmtcount}
\usepackage{placeins}
\usepackage{adjustbox}
\usepackage{tikz}
\usepackage{pgfplots}
\pgfplotsset{compat=1.18}
\usepackage{csvsimple}
\usepackage[top=1in, left=1in, right=1in, bottom=1in]{geometry}
\usepackage{nicefrac}
\usepackage{fancyhdr}
\usepackage{listings}
\usepackage{tabularx, booktabs, makecell}
\usepackage{hyperref,url}
\usepackage{listings}
\usepackage{mathtools} % For \xlongequal
\usepackage{siunitx}
\usepackage{multicol}
\usepackage{tcolorbox}
\usepackage[numbers]{natbib}
\usepackage{xfp}
\usepackage{circuitikz}
\usepackage{enumitem, enumerate}


\sisetup{
	round-mode=places,
	round-precision=4 % Set the number of decimal places
}


% Define a custom note environment
\newtcolorbox{note}{colback=lightgray!10!white, colframe=lightgray!50!black, title=Note}

\usepackage{titlesec}

\setcounter{secnumdepth}{4}

\titleformat{\paragraph}
{\normalfont\normalsize\bfseries}{\theparagraph}{1em}{}
\titlespacing*{\paragraph}
{0pt}{3.25ex plus 1ex minus .2ex}{1.5ex plus .2ex}




\makeatletter
\let\old@lstKV@SwitchCases\lstKV@SwitchCases
\def\lstKV@SwitchCases#1#2#3{}
\makeatother
\usepackage{lstlinebgrd}
\makeatletter
\let\lstKV@SwitchCases\old@lstKV@SwitchCases

\lst@Key{numbers}{none}{%
	\def\lst@PlaceNumber{\lst@linebgrd}%
	\lstKV@SwitchCases{#1}%
	{none:\\%
		left:\def\lst@PlaceNumber{\llap{\normalfont
				\lst@numberstyle{\thelstnumber}\kern\lst@numbersep}\lst@linebgrd}\\%
		right:\def\lst@PlaceNumber{\rlap{\normalfont
				\kern\linewidth \kern\lst@numbersep
				\lst@numberstyle{\thelstnumber}}\lst@linebgrd}%
	}{\PackageError{Listings}{Numbers #1 unknown}\@ehc}}
\makeatother
\newcounter{subListing}[subfigure]

\definecolor{codegreen}{rgb}{0,0.6,0}
\definecolor{codegray}{rgb}{0.5,0.5,0.5}
\definecolor{codepurple}{rgb}{0.58,0,0.82}
\definecolor{mygreen}{RGB}{28,172,0} 
\definecolor{mylilas}{RGB}{170,55,241}
\definecolor{backcolour}{rgb}{1,1,0.98}

\lstset{language=MATLAB,%
	backgroundcolor=\color{backcolour},   
	commentstyle=\color{codegreen},
	keywordstyle=\color{blue},
	numberstyle=\tiny\color{codegray},
	stringstyle=\color{codepurple},
	basicstyle=\tt\scriptsize,
	frame = LBtr,
	%frameround=T,
	rulecolor=\color{gray},
	showstringspaces=false,
	numbers=left,%
	numberstyle={\tiny\color{gray}},
	numbersep=8pt,
	breaklines=true,
	%postbreak=\mbox{\textcolor{yellow}{$\hookrightarrow$}\space},
	tabsize=2,
	escapechar=`,
	xleftmargin=1.8 em, 
	framexleftmargin=2em,
}

\newcommand*{\transpose}{{\mkern-1.5mu\mathsf{T}}}

\renewcommand{\baselinestretch}{1.5} 


\usepackage{titlesec}
\titleformat{\section}[block]
{\titlerule\addvspace{4pt}\normalfont\fontsize{14}{16}\bfseries}
{\thesection\enspace}{0pt}{}[\vspace{2pt}\titlerule]

\usepackage{xepersian}
\settextfont{XB Niloofar}
\setmathdigitfont{Yas}



\newcommand\question[1][\space]{
	\section[سوال \tartibi{section}]
	{سوال \tartibi{section}: #1}
}


\author{محمد رضیئی}
\title{گزارش تکلیف سری سوم}
\date{\today}


\definecolor{questioncolor}{rgb}{0.1, 0.1, 0.5}


\newcounter{rownum} % Define a counter for the row number
\newcounter{csvrownum} % Define a counter for the row number

\newcommand\saverread[2]{
	%	\(
	\csvreader[head=false, 
	before reading=\setcounter{csvrownum}{1}, after line=\stepcounter{csvrownum} 
	]{#1/results/saver.csv}{}%
	{\ifnum\thecsvrownum=#2 \num{\csvcoli} \fi}
	%	\)
}

\begin{document}
	
	
	% Set the page style to "fancy"...
	\pagestyle{fancy}
	%... then configure it.
	%		\fancyhead{} % clear all header fields
	%		\fancyhead[RO,LE]{\textbf{The performance of new graduates}}
	%		\fancyfoot{} % clear all footer fields
	%		\fancyfoot[LE,RO]{\thepage}
	%		\fancyfoot[LO,CE]{From: K. Grant}
	%		\fancyfoot[CO,RE]{To: Dean A. Smith}
	
\begin{titlepage}
	\begin{center}
		\vspace*{1cm}
		
		\includegraphics[width=0.3\textwidth]{HW/Sharif.jpg} % Replace with your logo
		
		\vspace{0.5cm}
		\textbf{\large دانشگاه صنعتی شریف}
		
		\vspace{0.5cm}
		{\Large \textbf{مهندسی برق}}
		
		
		\vspace{2.5cm}
		
		{\Large \textbf{گزارش تکلیف سری سوم}}
		

		
		\vspace{1cm}
		{\Huge \textbf{درس سیستم‌های آرایه فازی}}
		
		\vspace{2.5cm}
		
		\textbf{\large محمد رضیئی }
		
		\vfill
		

		

		
		\textbf{\large \today}
		
		\vspace{0.5cm}
	\end{center}
\end{titlepage}
	
	
	%%%%%%%%%%%%%%%%%%%%%%%%%%%%%%%%%%%%%%%%%%%%%%%%%%%%%%%%%%%%%%%%%%%%
	\FloatBarrier\question[]%1
	%%%%%%%%%%%%%%%%%%%%%%%%%%%%%%%%%%%%%%%%%%%%%%%%%%%%%%%%%%%%%%%%%%%%
	
	\begin{figure}[H]
		\centering
		\includegraphics[width=\linewidth]{Q1/results/ADS/circuit}
		\caption{}
		\label{fig:circuit}
	\end{figure}
	
	\begin{figure}[H]
		\centering
		\includegraphics[width=0.7\linewidth]{Q1/results/ADS/freq-range}
		\caption{}
		\label{fig:freq-range}
	\end{figure}
	
	
\begin{table}[h!]
	\centering
	\caption{Phase Data Table}
	\begin{tabular}{|l|c|c|c|c|}
		\hline
		Phase Type & Frequency (GHz) & Phase (deg) & VcQ & VcI \\
		\hline
		Maximum Phase & $60.00$ & $-91.810$ & $0.000$ & $2.000$ \\
		Minimum Phase & $60.00$ & $-178.187$ & $2.000$ & $0.000$ \\
		\hline
	\end{tabular}
\end{table}

\begin{equation}
	\text{frequency range} = (-91.810) - (-178.187) = \fpeval{(-91.810) - (-178.187)}^\circ \le 90^\circ 
\end{equation}


	
	
	\begin{figure}[H]
		\centering
		\includegraphics[width=0.7\linewidth]{Q1/results/ADS/s21-phase-vs-vci}
		\caption{}
		\label{fig:s21-phase-vs-vci}
	\end{figure}
	
	
	\begin{figure}[H]
		\centering
		\includegraphics[width=0.7\linewidth]{Q1/results/ADS/s21-phase-vs-vcq}
		\caption{}
		\label{fig:s21-phase-vs-vcq}
	\end{figure}
	
	
	\begin{figure}
		\centering
		\includegraphics[width=0.7\linewidth]{Q1/results/ADS/s21-phase-vs-vciq}
		\caption{}
		\label{fig:s21-phase-vs-vciq}
	\end{figure}









\begin{figure}[H]
	\begin{subfigure}{.45\linewidth}
		\centering
		\includegraphics[width=\linewidth]{Q1/results/phase-vs-voltage}
		\caption{}
		\label{fig:phase-vs-voltage}
	\end{subfigure}
	\hfill
	\begin{subfigure}{.45\linewidth}
		\centering
		\includegraphics[width=\linewidth]{Q1/results/amplitude-vs-voltage}
		\caption{}
		\label{fig:amplitude-vs-voltage}
	\end{subfigure}
	\caption{}
\end{figure}

\begin{figure}[H]
	\centering
	\includegraphics[width=0.7\linewidth]{Q1/results/vga-phase-map}
	\caption{}
	\label{fig:vga-phase-map}
\end{figure}
\begin{figure}[H]
	\centering
	\includegraphics[width=0.7\linewidth]{Q1/results/vga-phase-sort}
	\caption{}
	\label{fig:vga-phase-sort}
\end{figure}









\begin{figure}[H]
	\centering
	\includegraphics[width=0.7\linewidth]{Q1/results/vga-phase-map-nonideal}
	\caption{}
	\label{fig:vga-phase-map-nonideal}
\end{figure}
\begin{figure}[H]
	\centering
	\includegraphics[width=0.7\linewidth]{Q1/results/vga-phase-sort-nonideal}
	\caption{}
	\label{fig:vga-phase-sort-nonideal}
\end{figure}





	%%%%%%%%%%%%%%%%%%%%%%%%%%%%%%%%%%%%%%%%%%%%%%%%%%%%%%%%%%%%%%%%%%%%
\FloatBarrier\question[]%1
%%%%%%%%%%%%%%%%%%%%%%%%%%%%%%%%%%%%%%%%%%%%%%%%%%%%%%%%%%%%%%%%%%%%

\def\fo{30e9}% 30 GHz
\def\Cl{256e-15}% 256 fF



\begin{figure}[H]
	\centering
	\includegraphics[width=0.7\linewidth]{Q2/coupler}
	\caption{}
	\label{fig:coupler}
\end{figure}



%\begin{table}[h!]
%	\centering
%	\begin{tabular}{|c|c|}
%		\hline
%		\textbf{Parameter} & \textbf{Value} \\ \hline
%		H & $\num{8} \ \text{mil}$ \\ \hline
%		Er &$\num{3.38}$ \\ \hline
%		Mur & $\num{1}$ \\ \hline
%		Cond & $\num{58.7E6}$ \\ \hline
%		Hu & $\num{1e+36}$ \\ \hline
%		T & $\num{35} \ \text{$\mu$m}$ \\ \hline
%		TanD & $\num{0.0027}$ \\ \hline
%		FreqForEpsrTanD & $\num{10} \ \text{GHz}$ \\ \hline
%	\end{tabular}
%	\caption{Parameter Table for the Substrate}
%	\label{tab:parameters}
%\end{table}



\subsection{قسمت الف:}
ابتدا برای هر از بار های شکل 
	\ref{fig:reflective}
مقدار فاز $\Gamma$ را حساب می‌کنیم.


\begin{figure}[H]
	\centering
	\includegraphics[width=0.7\linewidth]{HW/reflective.png}
	\caption{}
	\label{fig:reflective}
\end{figure}

طبق راهنمایی تدریس یار درس، میزان پهنای باند برای بار‌های تماما موهومی بی‌نهایت است. از آنجایی که تمامی بارهای این شکل موهومی خالص است، از نظر پهنای باند برتری خاصی بایکدیگر ندارند. برتری آن ها در میزان درجه آزادی و می‌توان فاز را تغییر بدون این که مقادیر سلف و خازن نیاز باشد که منفی شود. بنابراین بهبود عملکرد این قسمت را در قسمت بعدی خواهیم دید:

\subsection{قسمت ب:}


اول از همه بگم که از سایت 
\url{https://leleivre.com/rf_lumped90hybrid.html}
روابط دقیق سلف و خازن های شکل زیر را در آوردم.


\begin{figure}[H]
	\centering
	\includegraphics[width=0.7\linewidth]{Q2/rf-lumped90hybrid}
	\caption{}
	\label{fig:rf-lumped90hybrid}
\end{figure}

که به صورت زیر است:
\begin{align}
	Z_s &= \frac{Z_0}{\sqrt{2}}, \\[10pt]
	L_1 &= \frac{Z_s}{2 \pi f_{0}}, \\[10pt]
	L_2 &= \frac{Z_0}{2 \pi f_{0}}, \\[10pt]
	C_1 &= \frac{1}{2 \pi f_{0}} \left( \frac{1}{Z_s} + \frac{1}{Z_0} \right).
\end{align}

حال با توجه به مقادیر سوال فرکانس کاری را بدست می‌آوریم:
\begin{equation}
	Z_0 = 50 \ \Omega
\end{equation}

\begin{equation}
	\implies \quad
	Z_s = \num{\fpeval{50 / sqrt(2)}} \ \Omega
\end{equation}


\begin{equation}
	\begin{cases}
		f_{0} = \frac{Z_s}{2 \pi L_1} \simeq \num{\fpeval{(50 / sqrt(2)) / (2 * pi * 93.8) * 1000}} \ \text{GHz}
		\\
		f_{0} = \frac{Z_0}{2 \pi L_2} \simeq \num{\fpeval{(50) / (2 * pi * 132) * 1000}} \ \text{GHz}
	\end{cases}
	\implies f_0 = 60 \ \text{GHz}
\end{equation}

متوجه میشویم که فرکانس باید ۶۰ گیگ باشد. برای این فرکانس مقادیر زیر را داریم:


\begin{equation}
	f_0 = 60 \ \text{GHz}
	\quad \implies C_1 = 128.1 \ \text{fF},\ \ L_1 = 93.78 \  \text{pH},\ \ L_2 = 132.6 \ \text{pH}
\end{equation}

در حالی که سوال برای ۳۰ گیگ میخواهد طراحی کند. بنابراین مقدار های مورد نظر به صورت زیر باید باشد:



\begin{equation}
	f_0 = 30 \ \text{GHz}
	\quad \implies C_1 = 256.2 \ \text{fF},\ \ L_1 = 187.6 \  \text{pH},\ \ L_2 = 265.3 \ \text{pH}
\end{equation}







ابتدا شکل \lr{a} را در نظر بگیرید:

\begin{center}
	\begin{circuitikz}
		% Node for the variable capacitor
		\node[circle, inner sep=1.3pt, fill=black] (start) at (0, 0) {};
		\node at (start)[xshift=-1cm] {(a):};
		\draw (start) to[variable capacitor, l_=$C_v$] (2, 0) 
		to[short] (2, 0) 
		node[ground, rotate=90]{};
	\end{circuitikz}
\end{center}

امپدانس آن برابر است با:

\begin{equation}
	(Z_L)_a = \frac{1}{j\omega C_v}
\end{equation}

حال گاما را حساب میکنیم:
\begin{equation}
	\Gamma_a = \frac{\frac{1}{j\omega C_v} - Z_0}{\frac{1}{j\omega C_v} + Z_0} =
	\frac{1- Z_0 \ {j\omega C_v} }{1 + Z_0 \ {j\omega C_v}}
\end{equation}



سپس فاز آن را در می‌آوریم:

\begin{align}
	\angle \Gamma_a &= \tan^{-1} (-Z_0 C_v \omega) - \tan^{-1}(Z_0 C_v \omega) 
	\\ &=
	- 2 \tan^{-1}(Z_0 C_v \omega)
\end{align}

می‌توان ثابت کرد که برای هر بار موهومی $Z_L$ مقدار فاز $\Gamma$ برابر است با:
\begin{equation}
	\angle\Gamma = -2 \tan\left(\frac{\mathcal{I}m\{Z_L\}}{Z_0}\right)
\end{equation}

باید شرط زیر برقرار باشد.
\begin{equation}
	0\ \le \ \tfrac{\pi}{2} + \angle \Gamma_a\ \le\  \tfrac{\pi}{2}
\end{equation}

در نتیجه داریم:

\begin{gather}
	-\tfrac{\pi}{4} \le\ -\tfrac{\pi}{4} + \tan^{-1}(Z_0 C_v \omega) \  \le\ 0  \\
	0 \le\ -\tfrac{\pi}{4} + \tan^{-1}(Z_0 C_v \omega) \  \le\  \tfrac\pi4 \\
	1 \le  Z_0 C_v \omega < \infty \\
	\frac{1}{Z_0 \omega}=\num{\fpeval{1/(2*pi*\fo*50)*1e15}} \le   C_v  < \infty \\
\end{gather}

از طرف دیگر کمینه مقدار $C_v$ برابر است با:
\begin{equation}
	C_{v, \min} = \frac{C_1}{\sqrt3} = \num{\fpeval{\Cl/sqrt(3)*1e15}}
\end{equation}


با این میزان $C_v$ نمی‌توان به $90^\circ$ اختلاف فاز رسید. با این حال شبیه سازی چنین می‌گوید:

\begin{figure}
	\centering
	\includegraphics[width=0.7\linewidth]{Q2/circuit-a}
	\caption{}
	\label{fig:circuit-a}
\end{figure}

%\begin{figure}[H]
%	\centering
%	\includegraphics[width=0.7\linewidth]{Q2/LC-a-phase-vs-cv}
%	\caption{}
%	\label{fig:lc-a-phase-vs-cv}
%\end{figure}

\begin{figure}[H]
	\centering
	\includegraphics[width=0.7\linewidth]{Q2/s21-vs-cv}
	\caption{}
	\label{fig:s21-vs-cv}
\end{figure}

که مقدار تغییرات فاز چنین است:


\begin{figure}[H]
	\centering
	\includegraphics[width=\linewidth]{Q2/s21-vs-cv-detail}
	\caption{}
	\label{fig:s21-vs-cv-detail}
\end{figure}

مشاهده می‌شود که با محدود کردن کمینه خازن متغیی، مقدار تغییرات فاز نیز کمتر شده است.



\begin{equation}
	\Gamma = \frac{Z_L - Z_0}{Z_L + Z_0}
\end{equation}




در مورد قسمت b نیز داریم:
\begin{center}
	\begin{circuitikz}
		% Node for the starting point
		\node[circle, inner sep=1.3pt, fill=black] (start) at (0, 0) {};
		\node at (start)[xshift=-1cm] {(b):};
		% Draw the variable capacitor
		\draw (start) 
		to[variable capacitor, l_=$C_v$] (2, 0)
		
		% Draw the inductor
		to[inductor, l_=$L$] (4, 0)
		
		% Draw the short and ground
		to[short] (4, 0)
		node[ground, rotate=90]{};
	\end{circuitikz}
\end{center}


\begin{equation}
	(Z_L)_b = \frac{1}{j\omega C_v} + j\omega L
\end{equation}

\begin{equation}
	\Gamma_b = \frac{\frac{1}{j\omega C_v} + j\omega L - Z_0}{\frac{1}{j\omega C_v} + j\omega L + Z_0} 
	=
	\frac{1 - \omega^2 L C_v - j \omega Z_0 C_v}{1 - \omega^2 L C_v + j \omega Z_0 C_v}
\end{equation}

\begin{align}
	\angle \Gamma_b =  \tan^{-1}\left( \frac{-\omega Z_0 C_v}{1 - \omega^2 L C_v} \right)  -  \tan^{-1}\left( \frac{\omega Z_0 C_v}{1 - \omega^2 L C_v} \right) = 2 \tan^{-1}\left( \frac{\omega Z_0 C_v}{\omega^2 L C_v - 1} \right)  
\end{align}

حال میخواهیم تغییرات فاز به اندازه ۹۰ درجه باشد:

\begin{equation}
	2\phi\ \le \ \angle \Gamma_b\ \le\  2\phi + \tfrac{\pi}{2}
\end{equation}

لذا داریم:

\begin{equation}
	\phi\ \le \ \tan^{-1}\left( \frac{\omega Z_0 C_v}{\omega^2 L C_v - 1} \right)  \ \le\  \phi + \tfrac{\pi}{4}
\end{equation}



%\begin{gather}
%	\tfrac{\pi}{4}  \le  \left.\tan^{-1}\left( \frac{\omega Z_0 C_v}{1 - \omega^2 L C_v} \right)\right.  \le \tfrac{\pi}{2} \\
%	1 \le  \left.\frac{\omega Z_0 C_v}{1 - \omega^2 L C_v }\right. \le \infty
%\end{gather}


\begin{equation}
	\omega Z_0 C_v = (\tan\phi)(\omega^2 L  C_v - 1)
	\quad\implies\quad
	 (C_v)_{\min} = \frac{1}{-\omega Z_0 \cot\phi + \omega^2 L } 
\end{equation}


\begin{equation}
	(C_v)_{\min} = \frac{C_1}{\sqrt{3}} = \num{\fpeval{1e15*\Cl/sqrt(3)}}
\end{equation}


\def\myphi{pi/4}
\begin{equation}
	L = \frac{1 + \omega Z_0 C_v \cot\phi}{\omega^2 C_v} = \frac{\sqrt{3}}{C_1 \omega^2} + \frac{Z_0\cot\phi}{\omega}
%	= 	\num{\fpeval{1e12*(sqrt(3)/(\Cl*((2*pi*\fo)^2)) + 50/(2*pi*\fo))}} \ \text{pH}
	= 	\num{\fpeval{1e12*(sqrt(3)/(\Cl*((2*pi*\fo)^2)) + (50*cot(\myphi))/(2*pi*\fo))}} \ \text{pH}
\end{equation}

بنابراین 
\begin{equation}
	\frac{1}{\omega^2L} = \num{\fpeval{1e15/(455.68e-12*(2*pi*\fo)^2)}} \ \text{fF} \le C_v \le \frac{C_1}{\sqrt3}
\end{equation}
%
%\begin{align}
%	\omega Z_0 C_v &= (\tan\phi)(\omega^2 L  C_v - 1)
%	\quad\implies\quad\\
%	(C_v)_{\max} &= \frac{1}{-\omega Z_0 \cot(\phi+\tfrac{\pi}4) + \omega^2 L } 
%	\\&= \frac{1}{\omega Z_0 (\cot(\phi) - \cot(\phi+\tfrac{\pi}4)) + \frac{\sqrt{3}}{C1}} =
%	\num{\fpeval{1e15/(2*pi*\fo*(cot(\myphi)-cot(\myphi+pi/4))+sqrt(3)/\Cl)}}
%\end{align}

\begin{figure}[H]
	\centering
	\includegraphics[width=0.7\linewidth]{Q2/circuit-b}
	\caption{}
	\label{fig:circuit-b}
\end{figure}

\begin{figure}[H]
	\centering
	\includegraphics[width=0.7\linewidth]{Q2/LC-b-phase-vs-cv}
	\caption{}
	\label{fig:lc-b-phase-vs-cv}
\end{figure}

\begin{figure}[H]
	\centering
	\includegraphics[width=\linewidth]{Q2/s21-vs-cv-detail-b}
	\caption{}
	\label{fig:s21-vs-cv-detail-b}
\end{figure}


حال شکل \lr{c} رو در نظر می‌گیریم:

\begin{center}
	\begin{circuitikz}
		% Node for the starting point
		\node[circle, inner sep=1.3pt, fill=black] (start) at (-.5, 0) {};
		\node at (start)[xshift=-1cm] {(c):};
		% Draw the variable capacitor
		\draw (start) 
		to[variable capacitor, l_=$C_v$] (3, 0)
		
		% Draw the inductor
		to[inductor, l_=$L$] (5, 0)
		
		% Draw the short and ground
		to[short] (5, 0)
		node[ground, rotate=90]{};
		
		\draw (0, 0) 
		to[capacitor, l_=$C_T$] (0, -2)
		
		% Draw the short and ground
		to[short] (0, -2)
		node[ground]{};
		
	\end{circuitikz}
\end{center}


\begin{equation}
	j(\omega L - \frac{1}{C_v\omega}) \| \frac{1}{j C_T\omega} = \frac{j}{C_T\omega + \frac{C_v\omega}{C_v\omega^2L - 1}}  
\end{equation}

مخرج را مساوی صفر قرار می‌دهیم:
\begin{equation}
	C_T\omega = \frac{C_v\omega}{1 - C_v\omega^2 L}
	\quad\implies\quad
	1 - C_v\omega^2 L = \frac{C_v}{C_T}
	\quad\implies\quad
	C_v = \frac{1}{\omega^2 L+1/C_T}
\end{equation}

حال قسمت موهومی آن را با $Z_0$ مساوی قرار می‌دهیم:

\begin{gather}
	C_T\omega + \frac{C_v\omega}{C_v\omega^2 L - 1} = \frac{1}{Z_0} \\
	C_v\omega = (C_v\omega^2L-1)(1/Z_0 -C_T\omega)
%	\\ C_v(\omega - \omega^2L/Z_0)
\end{gather}

مقادیر سلف و خازن در 
\begin{figure}[H]
	\centering
	\includegraphics[width=0.7\linewidth]{Q2/circuit-c}
	\caption{}
	\label{fig:circuit-c}
\end{figure}


\begin{figure}[H]
	\centering
	\includegraphics[width=0.7\linewidth]{Q2/s21-vs-cv-detail-c}
	\caption{}
	\label{fig:s21-vs-cv-detail-c}
\end{figure}


\begin{figure}[H]
	\centering
	\includegraphics[width=0.7\linewidth]{Q2/s21-vs-cv-c}
	\caption{}
	\label{fig:s21-vs-cv-c}
\end{figure}

حال قسمت d را در نظر بگیرید.
\begin{center}
	\begin{circuitikz}
		% Node for the starting point
		\node[circle, inner sep=1.3pt, fill=black] (start) at (-.5, 0) {};
		\node at (start)[xshift=-1cm] {(b):};
		% Draw the variable capacitor
		\draw (start) 
		to[variable capacitor, l_=$C_v$] (2, 0)
		
		% Draw the inductor
		to[inductor, l_=$L$] (4, 0)
		
		% Draw the short and ground
		to[short] (4, 0)
		node[ground, rotate=90]{};
		
		\draw (0,0) 
		to[short] (0, -2)
		to[variable capacitor, l_=$C_v$] (2, -2)
		
		% Draw the inductor
		to[inductor, l_=$L$] (4, -2)
		
		% Draw the short and ground
		to[short] (4, -2)
		node[ground, rotate=90]{};
	\end{circuitikz}
\end{center}

این در حقیقت همان قسمت b است با تغییر پارامتر است که مقادیر سلف آن نصف شده است.

با شبیه سازی داریم:

\begin{figure}[H]
	\centering
	\includegraphics[width=0.7\linewidth]{Q2/circuit-d}
	\caption{}
	\label{fig:circuit-d}
\end{figure}

\begin{figure}[H]
	\centering
	\includegraphics[width=0.7\linewidth]{Q2/circuit-d1}
	\caption{}
	\label{fig:circuit-d1}
\end{figure}

\begin{figure}[H]
	\centering
	\includegraphics[width=0.7\linewidth]{Q2/circuit-d2}
	\caption{}
	\label{fig:circuit-d2}
\end{figure}



نتایج برای مدار distribution نیز یکسان است منتها 90 درجه بیشتر است. مدار آن به صورت زیر خواهد بود.



\begin{figure}[H]
	\centering
	\includegraphics[width=0.7\linewidth]{Q2/circuit-dist}
	\caption{}
	\label{fig:circuit-dist}
\end{figure}

که مقادیر زیر لایه آن ها در زیر است.

\begin{table}[h!]
	\centering
	\begin{tabular}{|c|c|}
			\hline
			\textbf{Parameter} & \textbf{Value} \\ \hline
			H & $\num{8} \ \text{mil}$ \\ \hline
			Er &$\num{3.38}$ \\ \hline
			Mur & $\num{1}$ \\ \hline
			Cond & $\num{58.7E6}$ \\ \hline
			Hu & $\num{1e+36}$ \\ \hline
			T & $\num{35} \ \text{$\mu$m}$ \\ \hline
			TanD & $\num{0.0027}$ \\ \hline
			FreqForEpsrTanD & $\num{10} \ \text{GHz}$ \\ \hline
		\end{tabular}
	\caption{\lr{Parameter Table for the Substrate}}
	\label{tab:parameters}
\end{table}


	%%%%%%%%%%%%%%%%%%%%%%%%%%%%%%%%%%%%%%%%%%%%%%%%%%%%%%%%%%%%%%%%%%%%
\FloatBarrier\question[\lr{PHASE SHIFTERS IN LITERATURE}]%1
%%%%%%%%%%%%%%%%%%%%%%%%%%%%%%%%%%%%%%%%%%%%%%%%%%%%%%%%%%%%%%%%%%%%
\begin{figure}[H]
	\centering
	\includegraphics[width=0.7\linewidth]{Q2/circuit-d4}
	\caption{}
	\label{fig:circuit-d4}
\end{figure}

طبق شکل بالا، امپدانس ورودی برابر است با:

\begin{equation}
	Z_\text{in}(j\omega) = L_2 j\omega + \left( \frac{1}{C_v j\omega} + L_1 j\omega \right) \parallel \frac{1}{C_1 j\omega}
\end{equation}


\begin{equation}
	\rightarrow Z_\text{in}(j\omega) = j \left( L_2 \omega + \frac{C_v L_1 \omega^2 - 1}{(C_1 + C_v) \omega - C_1 C_v L_1 \omega^3} \right)
\end{equation}

بنابراین، با استفاده از فرمول شیفت فاز که قبلاً به دست آورده‌ایم، داریم:

\begin{equation}
	\phi = -\frac{\pi}{2} - 2 \tan^{-1} \left( \frac{\mathcal{I}m \{Z_\text{in}\}}{Z_0} \right)
\end{equation}

با جایگذاری مقاومت ورودی در فرمول شیفت فاز، فرمول بعد از ساده‌سازی به صورت زیر درمی‌آید:

\begin{equation}
	\phi = -\frac{\pi}{2} - 2 \tan^{-1} \left( L_2 \omega + \frac{1}{Z_0} \cdot \frac{C_v L_1 \omega^2 - 1}{(C_1 + C_v) \omega - C_1 C_v L_1 \omega^3} \right)
\end{equation}




در طراحی \lr{Hybrid Coupler} برای فرکانس مشخص، مقادیر پارامترهای کلیدی مدار به‌صورت زیر تعیین می‌شوند:


\begin{equation}
	f_0 = 60 \ \text{GHz}
	\quad \implies C_1 = 128.1 \ \text{fF},\ \ L_1 = 93.78 \  \text{pH},\ \ L_2 = 132.6 \ \text{pH}
\end{equation}


سپس برای ساده‌سازی طراحی اجزا، فرض می‌شود که مقدار \( L_2 \) برابر صفر در نظر گرفته شود. در این حالت، فرمول شیفت فاز به شکل زیر ساده می‌شود:

\begin{equation}
	\phi = -\frac{\pi}{2} - 2 \tan^{-1} \left( \omega \frac{1}{Z_0} \cdot \frac{C_v L_1 \omega^2 - 1}{(C_1 + C_v) \omega - C_1 C_v L_1 \omega^3} \right)
\end{equation}



برای ادامه طراحی، مقادیر اولیه برای سلف \( L_1 \) و خازن \( C_1 \) به ترتیب زیر انتخاب می‌شوند:

\[
L_1 = 100 \, \text{pH}, \quad C_1 = 100 \, \text{fF}.
\]



برای اطمینان از عملکرد مناسب مدار، محدوده خازن متغیر \( C_v \) به صورت زیر تعیین می‌شود:

\begin{equation}
	C_{v,\min} = \frac{1}{\omega^2 L_1} \approx 70 \, \text{fF}
\end{equation}

\begin{equation}
	C_{v,\max} = \frac{C_1}{\omega^2 L_1 C_1 - 1} \approx 237 \, \text{fF}
\end{equation}



با استفاده از این مقادیر و روابط به‌دست‌آمده، تمام شیفت فازهای موردنیاز در بازه \( 0^\circ \) تا \( 180^\circ \) امکان‌پذیر است. این مدار کامل شیفت‌دهنده فاز به‌علاوه سایر اجزای مدار \lr{Hybrid Coupler} در شکل زیر نمایش داده شده است.


\begin{figure}[H]
	\centering
	\includegraphics[width=0.7\linewidth]{Q2/circuit-d3}
	\caption{}
	\label{fig:circuit-d3}
\end{figure}

\begin{figure}[H]
	\centering
	\includegraphics[width=0.7\linewidth]{Q2/LC-a-phase-vs-cv}
	\caption{}
	\label{fig:lc-a-phase-vs-cv}
\end{figure}



\subsection*{مزایا و معایب \lr{Hybrid Coupler}}

\lr{Hybrid Coupler} به‌عنوان یک ابزار کلیدی در سیستم‌های مایکروویو و مخابراتی، با استفاده از خطوط انتقال و عناصر توزیع‌شده، مزایای متعددی را ارائه می‌دهد. این مدار نه‌تنها باعث کاهش مساحت اشغال‌شده می‌شود، بلکه پهنای باند وسیع‌تری نسبت به روش‌های سنتی فراهم می‌کند. بااین‌حال، استفاده از آن نیز محدودیت‌ها و چالش‌هایی دارد که باید در طراحی موردتوجه قرار گیرد.

\noindent مزایای \lr{Hybrid Coupler} عبارت‌اند از:
\begin{itemize}
	\item کاهش مساحت کلی موردنیاز به‌واسطه حذف خطوط طولانی \(\lambda/4\) که باعث صرفه‌جویی در فضای فیزیکی مدار می‌شود.
	\item ارائه پهنای باند وسیع‌تر نسبت به \lr{Branch Line Coupler} که عملکرد مدار را در بازه وسیع‌تری از فرکانس تضمین می‌کند.
	\item کاهش تلفات انتقال سیگنال به دلیل کاهش طول خطوط و مقاومت داخلی آن‌ها.
	\item افزایش انعطاف‌پذیری طراحی با امکان استفاده از خازن‌ها و القاگرهای موازی، که درجه آزادی بیشتری برای تنظیم مدار فراهم می‌کند.
	\item کاهش اندازه ترانسفورمرها با استفاده از خازن‌های توزیع‌شده (\(C_c\)) که بهینه‌سازی فضای موردنیاز را تسهیل می‌کند.
	\item بهبود امپدانس و تطبیق بهتر با سیگنال‌های ورودی و خروجی، که منجر به کارایی بیشتر مدار می‌شود.
	\item تقویت ایزولاسیون میان خطوط و کاهش تداخل سیگنال که در سیستم‌های حساس مخابراتی بسیار مهم است.
\end{itemize}

\noindent با وجود این مزایا، معایب زیر نیز وجود دارد:
\begin{itemize}
	\item پیچیدگی در طراحی به دلیل نیاز به تنظیم دقیق فاصله و طول خطوط که می‌تواند زمان و هزینه طراحی را افزایش دهد.
	\item افزایش نیاز به خازن‌های با ظرفیت بالا برای بهینه‌سازی عملکرد مدار، که ممکن است هزینه‌های اضافی را به همراه داشته باشد.
\end{itemize}

\noindent در مجموع، \lr{Hybrid Coupler} به دلیل انعطاف‌پذیری و کارایی بالا، یکی از ابزارهای مؤثر در طراحی مدارهای مایکروویو است، اما برای دستیابی به بهترین عملکرد، نیاز به دقت بالا و توجه به محدودیت‌های طراحی وجود دارد.



\subsection*{تحلیل مدار Coupler در حالت متوالی (Cascade)}

در این بخش به بررسی تأثیر اتصال متوالی \lr{Coupler} بر شیفت فاز، تلفات، پهنای باند و تأخیر گروهی پرداخته می‌شود.

\noindent\textbf{شیفت فاز}
در اتصال متوالی، میزان شیفت فاز کلی برابر با جمع شیفت فازهای هر \lr{Coupler} است. فرمول شیفت فاز به صورت زیر بیان می‌شود:
\begin{equation}
	\phi = -\frac{\pi}{2} - 2 \tan^{-1} \left( \frac{\mathcal{I}m \{Z_L\}}{Z_0} \right)
\end{equation}
برای \(n\) کوپلر متوالی، شیفت فاز کلی (\(\phi_\text{tot}\)) به صورت زیر محاسبه می‌شود:
\begin{equation}
	\phi_\text{tot} = \sum_{i=1}^{n} \Delta \phi_i = n \left( -\frac{\pi}{2} + 2 \tan^{-1} \left( \frac{\mathcal{I}m \{Z_L\}}{Z_0} \right) \right)
\end{equation}

\noindent\textbf{تلفات}
تلفات کلی برای \(n\) طبقه کوپلر برابر است با:
\begin{equation}
	L_\text{Cascade} = 2 \cdot n \cdot L_\text{single} + n
\end{equation}
تلفات کلی با افزایش تعداد طبقات به دلیل اجزای اضافی در مدار افزایش می‌یابد.

\noindent\textbf{پهنای باند}
پهنای باند کلی (\(BW_\text{tot}\)) در اتصال متوالی برابر است با کمترین پهنای باند از طبقات منفرد:
\begin{equation}
	BW_\text{tot} = \min (BW_\text{single})
\end{equation}
این موضوع نشان می‌دهد که افزایش تعداد طبقات می‌تواند پهنای باند کلی سیستم را کاهش دهد.

\noindent\textbf{تأخیر گروهی (\lr{Group Delay})}
با افزایش تعداد طبقات، تأخیر گروهی کلی (\(delay_\text{tot}\)) به صورت زیر محاسبه می‌شود:
\begin{equation}
	delay_\text{tot} = n \cdot delay_\text{single}
\end{equation}


اتصال متوالی \lr{Coupler} می‌تواند شیفت فاز را به طور خطی افزایش دهد، اما همراه با تلفات، کاهش پهنای باند و افزایش تأخیر گروهی است. بنابراین، طراحی بهینه باید به نحوی انجام شود که تأثیرات منفی به حداقل برسد.





	%%%%%%%%%%%%%%%%%%%%%%%%%%%%%%%%%%%%%%%%%%%%%%%%%%%%%%%%%%%%%%%%%%%%
	\newpage
	\bibliographystyle{plainnat}
	\nocite{*}
	\bibliography{references}
	
\end{document}