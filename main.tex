\documentclass[12pt,onecolumn,a4paper]{article}
\usepackage{amsthm,amsmath,amssymb,bm}
\usepackage{epsfig,graphicx,subcaption}
\usepackage{float}
\usepackage{color,xcolor}
\usepackage{fmtcount}
\usepackage{placeins}
\usepackage{adjustbox}
\usepackage{tikz}
\usepackage{pgfplots}
\pgfplotsset{compat=1.18}
\usepackage{csvsimple}
\usepackage[top=1in, left=1in, right=1in, bottom=1in]{geometry}
\usepackage{nicefrac}
\usepackage{fancyhdr}
\usepackage{listings}
\usepackage{tabularx, booktabs, makecell}
\usepackage{hyperref,url}
\usepackage{listings}
\usepackage{mathtools} % For \xlongequal
\usepackage{siunitx}
\usepackage{multicol}
\usepackage{tcolorbox}
\usepackage[numbers]{natbib}
\usepackage{pgffor}
\usepackage{pgfmath}
\usepackage{colortbl}

% Define a custom note environment
\newtcolorbox{note}{colback=lightgray!10!white, colframe=lightgray!50!black, title=Note}

\usepackage{titlesec}

\setcounter{secnumdepth}{4}

\titleformat{\paragraph}
{\normalfont\normalsize\bfseries}{\theparagraph}{1em}{}
\titlespacing*{\paragraph}
{0pt}{3.25ex plus 1ex minus .2ex}{1.5ex plus .2ex}




\makeatletter
\let\old@lstKV@SwitchCases\lstKV@SwitchCases
\def\lstKV@SwitchCases#1#2#3{}
\makeatother
\usepackage{lstlinebgrd}
\makeatletter
\let\lstKV@SwitchCases\old@lstKV@SwitchCases

\lst@Key{numbers}{none}{%
	\def\lst@PlaceNumber{\lst@linebgrd}%
	\lstKV@SwitchCases{#1}%
	{none:\\%
		left:\def\lst@PlaceNumber{\llap{\normalfont
				\lst@numberstyle{\thelstnumber}\kern\lst@numbersep}\lst@linebgrd}\\%
		right:\def\lst@PlaceNumber{\rlap{\normalfont
				\kern\linewidth \kern\lst@numbersep
				\lst@numberstyle{\thelstnumber}}\lst@linebgrd}%
	}{\PackageError{Listings}{Numbers #1 unknown}\@ehc}}
\makeatother
\newcounter{subListing}[subfigure]

\definecolor{codegreen}{rgb}{0,0.6,0}
\definecolor{codegray}{rgb}{0.5,0.5,0.5}
\definecolor{codepurple}{rgb}{0.58,0,0.82}
\definecolor{mygreen}{RGB}{28,172,0} 
\definecolor{mylilas}{RGB}{170,55,241}
\definecolor{backcolour}{rgb}{1,1,0.98}

\lstset{language=MATLAB,%
	backgroundcolor=\color{backcolour},   
	commentstyle=\color{codegreen},
	keywordstyle=\color{blue},
	numberstyle=\tiny\color{codegray},
	stringstyle=\color{codepurple},
	basicstyle=\tt\scriptsize,
	frame = LBtr,
	%frameround=T,
	rulecolor=\color{gray},
	showstringspaces=false,
	numbers=left,%
	numberstyle={\tiny\color{gray}},
	numbersep=8pt,
	breaklines=true,
	%postbreak=\mbox{\textcolor{yellow}{$\hookrightarrow$}\space},
	tabsize=2,
	escapechar=`,
	xleftmargin=1.8 em, 
	framexleftmargin=2em,
}

\newcommand*{\transpose}{{\mkern-1.5mu\mathsf{T}}}


\usepackage{titlesec}
\titleformat{\section}[block]
{\titlerule\addvspace{4pt}\normalfont\fontsize{14}{16}\bfseries}
{\thesection\enspace}{0pt}{}[\vspace{2pt}\titlerule]


\usepackage{xepersian}
\settextfont{XB Niloofar}
\setmathdigitfont{Yas}


\newcommand\question[1][\space]{
	\section[سوال \tartibi{section}]
	{سوال \tartibi{section}: #1}
}


\author{محمد رضیئی فیجانی}
\title{امتحان میانترم درس آرایه فازی}
\date{\today}


\definecolor{questioncolor}{rgb}{0.1, 0.1, 0.5}


\newcounter{rownum} % Define a counter for the row number
\newcounter{csvrownum} % Define a counter for the row number

\newcommand\saverread[2]{
	%	\(
	\csvreader[head=false, 
	before reading=\setcounter{csvrownum}{1}, after line=\stepcounter{csvrownum} 
	]{#1/results/saver.csv}{}%
	{\ifnum\thecsvrownum=#2 \num{\csvcoli} \fi}
	%	\)
}



\begin{document}
	
	
	% Set the page style to "fancy"...
	\pagestyle{fancy}
	%... then configure it.
	%		\fancyhead{} % clear all header fields
	%		\fancyhead[RO,LE]{\textbf{The performance of new graduates}}
	%		\fancyfoot{} % clear all footer fields
	%		\fancyfoot[LE,RO]{\thepage}
	%		\fancyfoot[LO,CE]{From: K. Grant}
	%		\fancyfoot[CO,RE]{To: Dean A. Smith}
	\maketitle
	
	
	
	%%%%%%%%%%%%%%%%%%%%%%%%%%%%%%%%%%%%%%%%%%%%%%%%%%%%%%%%%%%%%%%%%%%%
	\FloatBarrier\question[\lr{Passive and Active Reconfigurable Intelligent Surface}]%1
	%%%%%%%%%%%%%%%%%%%%%%%%%%%%%%%%%%%%%%%%%%%%%%%%%%%%%%%%%%%%%%%%%%%%
	
	

	توان دریافتی از یک \lr{RIS} غیرفعال توسط گیرنده به صورت زیر مدل می‌شود:
	
	\begin{equation}
		P_r = \frac{N^2 P_T}{16L}
	\end{equation}
	
	
	\begin{equation}
		\lambda = \frac{c}{f} = \frac{\num{3e8}}{\num{5.8e9}} = 0.0517 \ \text{m} 
	\end{equation}
	
	
	
	
				
	\begin{align}
		\text{SNR} &= P_t \cdot G_t \cdot G_r \cdot \left( \frac{\lambda}{4 \pi \ell} \right)^n \cdot \frac{1}{K_B \cdot T_0 \cdot B_w \cdot F \cdot \mathcal{L}}
	\end{align}
	
	که در آن:
	\begin{latin}
		\begin{itemize}
			\item \(\lambda\): Carrier wavelength,
			\item \(\ell\): Link range,
			\item \(K_B = 1.38 \times 10^{-23} \, \text{J/K}\): Boltzmann constant,
			\item \(T_0 = 290 \, \text{K}\): Absolute room temperature,
			\item \(B_w\): Channel bandwidth,
			\item \(F\): Receiver noise figure,
			\item \(\mathcal{L}\): Losses in the system.
		\end{itemize}
	\end{latin}
	
	
	
	
	%%%%%%%%%%%%%%%%%%%%%%%%%%%%%%%%%%%%%%%%%%%%%%%%%%%%%%%%%%%%%%%%%%%%
	\FloatBarrier\question[\lr{Array Pattern Synthesis}]%1
	%%%%%%%%%%%%%%%%%%%%%%%%%%%%%%%%%%%%%%%%%%%%%%%%%%%%%%%%%%%%%%%%%%%%
	
	
	تابع 	پنجره همینگ N تایی  به صورت زیر است.
	
	\begin{latin}
		\lstinputlisting{Q2/hamming_window.m}
	\end{latin}
خروجی آن را برای $N=21$ رسم می‌کنیم.
	
\begin{figure}[H]
	\centering
	\includegraphics[width=.5\linewidth]{Q2/results/hamming-21}
	\caption{}
	\label{fig:hamming-21}
\end{figure}

ابتدا AF را بدون tapering رسم می‌کنیم.
	
	
\begin{figure}[H]
	\centering
	\includegraphics[width=\linewidth]{Q2/results/af}
	\caption{}
	\label{fig:af}
\end{figure}
	
\begin{figure}[H]
	\centering
	\includegraphics[width=\linewidth]{Q2/results/af-db}
	\caption{}
	\label{fig:af-db}
\end{figure}
	
	سپس پنجره hamming را به آن اعمال می‌کنیم:
	
\begin{figure}[H]
	\centering
	\includegraphics[width=\linewidth]{Q2/results/af-with-hamming}
	\caption{}
	\label{fig:af-with-hamming}
\end{figure}
\begin{figure}[H]
	\centering
	\includegraphics[width=\linewidth]{Q2/results/af-db-with-hamming}
	\caption{}
	\label{fig:af-db-with-hamming}
\end{figure}

اطلاعات خواسته شده در شکل مشخص شده است.


برای آرایه خطی مورد های زیر را داریم:

\begin{figure}[H]
	\centering
	\includegraphics[width=0.5\linewidth]{Q2/results/a-cart}
	\caption{}
	\label{fig:a-cart}
\end{figure}
\begin{figure}[H]
	\centering
	\includegraphics[width=0.5\linewidth]{Q2/results/a-imagesc}
	\caption{}
	\label{fig:a-imagesc}
\end{figure}
\begin{figure}[H]
	\centering
	\includegraphics[width=.51\linewidth]{Q2/results/a-p-plot3}
	\caption{}
	\label{fig:a-p-plot3}
\end{figure}
\begin{figure}[H]
	\centering
	\includegraphics[width=\linewidth]{Q2/results/a-phi-0-mag}
	\caption{}
	\label{fig:a-phi-0-mag}
\end{figure}
\begin{figure}[H]
	\centering
	\includegraphics[width=\linewidth]{Q2/results/a-phi-0-db}
	\caption{}
	\label{fig:a-phi-0-db}
\end{figure}

و برای دایروی نیز موارد زیر را داریم:


\begin{figure}[H]
	\centering
	\includegraphics[width=0.5\linewidth]{Q2/results/b-cart}
	\caption{}
	\label{fig:b-cart}
\end{figure}
\begin{figure}[H]
	\centering
	\includegraphics[width=0.5\linewidth]{Q2/results/b-imagesc}
	\caption{}
	\label{fig:b-imagesc}
\end{figure}
\begin{figure}[H]
	\centering
	\includegraphics[width=.51\linewidth]{Q2/results/b-p-plot3}
	\caption{}
	\label{fig:b-p-plot3}
\end{figure}
\begin{figure}[H]
	\centering
	\includegraphics[width=\linewidth]{Q2/results/b-phi-0-mag}
	\caption{}
	\label{fig:b-phi-0-mag}
\end{figure}
\begin{figure}[H]
	\centering
	\includegraphics[width=\linewidth]{Q2/results/b-phi-0-db}
	\caption{}
	\label{fig:b-phi-0-db}
\end{figure}





	
	
	
	%%%%%%%%%%%%%%%%%%%%%%%%%%%%%%%%%%%%%%%%%%%%%%%%%%%%%%%%%%%%%%%%%%%%
	\FloatBarrier\question[\lr{Two dimensional array}]%3
	%%%%%%%%%%%%%%%%%%%%%%%%%%%%%%%%%%%%%%%%%%%%%%%%%%%%%%%%%%%%%%%%%%%%
	
	





	
	برای آرایه مربعی با \(N = 4\) (تعداد 16 آنتن) و فاصله المان‌ها \(d = 0.6\lambda\)، وزن‌های آرایه برای چرخاندن پرتو اصلی به زوایای \(\theta = 30^\circ\) و \(\phi = 45^\circ\) به صورت زیر محاسبه می‌شوند:
	

	زاویه فاز برای هر المان در موقعیت \((m, n)\) به صورت زیر است:
	\[
	\psi_{m,n} = -2\pi \left[m \sin\theta \cos\phi \ \frac{d_x}{\lambda} + n \sin\theta \sin\phi\ \frac{d_y}{\lambda} \right]
	\]
	
		\begin{equation}
		\theta = 30
		\quad,\quad
		\phi = 45
	\end{equation}

	
	
	جایگذاری مقادیر:
	\[
	\sin\theta = 0.5, \quad \cos\phi = \sin\phi = \frac{\sqrt{2}}{2}
	\]
	
		\begin{itemize}
		\item \(x\):
		\[
		\sin\theta \cos\phi =  \tfrac{\sqrt{2}}{4}
		\]
		\item \(y\):
		\[
		\sin\theta \sin\phi = \tfrac{\sqrt{2}}{4}
		\]
	\end{itemize}
	
	\[
	\psi_{m,n} = -0.4243\pi (m + n)
	\]
	
	وزن‌های آرایه برای هر المان:
	\[
	w_{m,n} = e^{-j 0.4243\pi (m + n)}
	\]
	

	وزن‌ها در ماتریس \(4 \times 4\) را میتوان به صورت ماتریسی نوشت.
	
%\[
%W =
%\begin{bmatrix}
%	\foreach \m in {0,1,2,3} {
%		\foreach \n  in {0,1,2,3} {
%      		\pgfmathparse{\m + \n}%
%			\pgfmathtruncatemacro{\sum}{\pgfmathresult} % Convert to integer
%			\pgfmathsetmacro{\p}{0.4243 * (\m + \n)}%
%			\ifnum\sum=0 1%
%			\else e^{-j (\p) \pi }%
%			\fi%
%			\ifnum\n<3 \quad \else \\ \fi
%		}
%	}
%	\end{bmatrix}
%\]



\begin{equation}
	W = 
	\left[
	\raisebox{-1.7cm}{
		\begin{tikzpicture}[node distance=1cm, every node/.style={anchor=center}]
			\def\dr{2.5}
			\def\dc{1}
			\foreach \m in {0,1,2,3} {
				\foreach \n in {0,1,2,3} {
					\pgfmathparse{\m + \n}%
					\pgfmathtruncatemacro{\sum}{\pgfmathresult} % Convert to integer
					\pgfmathsetmacro{\p}{0.4243 * (\m + \n)}%
					% Calculate element and place in matrix
					\ifnum\sum=0
					\node at (\n*\dr,-\m*\dc) {$1$};
					\else
					\node at (\n*\dr,-\m*\dc) {$e^{-j (\p) \pi}$};
					\fi
				}
			}
	\end{tikzpicture}}
	\right]
\end{equation}
	
	حال که ماترس را ساختیم مانند tapering با آن برخورد می‌کنیم که تنها فاز را عوض می‌کند و دامنه را تغییر نمی‌دهد.
	ابتدا مسیله را بدون ماتریس های وزن حل میکنیم.
	
	
	
	
	
	
\begin{figure}[H]
	\centering
	\includegraphics[width=0.7\linewidth]{Q3/results/c-p-plot3}
	\caption{}
	\label{fig:c-p-plot3}
\end{figure}

\begin{figure}[H]
	\centering
	\includegraphics[width=.5\linewidth]{Q3/results/c-imagesc}
	\caption{}
	\label{fig:c-imagesc}
\end{figure}

\begin{figure}[H]
	\centering
	\includegraphics[width=0.7\linewidth]{Q3/results/c-cart}
	\caption{}
	\label{fig:c-cart}
\end{figure}


حال اگر $\phi = 24^\circ$ فرض کنیم، داریم:
\begin{figure}[H]
	\centering
	\includegraphics[width=0.7\linewidth]{Q3/results/c-phi-0-mag}
	\caption{}
	\label{fig:c-phi-0-mag}
\end{figure}
\begin{figure}[H]
	\centering
	\includegraphics[width=\linewidth]{Q3/results/c-phi-0-db}
	\caption{}
	\label{fig:c-phi-0-db}
\end{figure}



%	\[
%	W =
%	\begin{bmatrix}
%		e^{j0} & e^{-j0.4243\pi} & e^{-j1.4243\pi} & e^{-j2.4243\pi} \\
%		e^{-j1.333\pi} & e^{-j2.666\pi} & e^{-j4\pi} & e^{-j5.333\pi} \\
%		e^{-j2.666\pi} & e^{-j4\pi} & e^{-j5.333\pi} & e^{-j6.666\pi} \\
%		e^{-j4\pi} & e^{-j5.333\pi} & e^{-j6.666\pi} & e^{-j8\pi}
%	\end{bmatrix}
%	\]
%	
	
	

	
	
سپس وزن ها را به آن اعمال می‌کنیم:
\begin{figure}[H]
	\centering
	\includegraphics[width=0.5\linewidth]{Q3/results/w-p-plot3}
	\caption{}
	\label{fig:w-p-plot3}
\end{figure}

\begin{figure}[H]
	\centering
	\includegraphics[width=0.5\linewidth]{Q3/results/w-imagesc}
	\caption{}
	\label{fig:w-imagesc}
\end{figure}

\begin{figure}[H]
	\centering
	\includegraphics[width=0.5\linewidth]{Q3/results/w-cart}
	\caption{}
	\label{fig:w-cart}
\end{figure}



چون طبق فرض بیشینه مقدار گین در $\phi = 24^\circ$ رخ می‌دهد برای این زاویه داریم:

\begin{figure}[H]
	\centering
	\includegraphics[width=\linewidth]{Q3/results/w-phi-0-mag}
	\caption{}
	\label{fig:w-phi-0-mag}
\end{figure}
\begin{figure}[H]
	\centering
	\includegraphics[width=\linewidth]{Q3/results/w-phi-0-db}
	\caption{}
	\label{fig:w-phi-0-db}
\end{figure}

		

	اگر فرض کنیم آنتن ها به صورت ایزوتروپیک هستند، یعنی پترن کلی آنتن همین فاکتور آرایه خواهد بود. بنابراین از روی شکل های بالا می‌توان جدول زیر را تشکیل داد.
	
\begin{table}[H]
	\centering
	\begin{tabular}{ccc}
		\toprule\rowcolor{gray!10}
		\textbf{HPBW} & \textbf{NNBW} & \textbf{Gain Peak}\\\midrule
		$0.142\pi$ & $0.47\pi$ & $16 = 24 \ \text{dB}$ \\
		\bottomrule
	\end{tabular}
\end{table}


حال برای قسمت آخر به صورت زیر انجام می‌دهیم:


ابتدا ماتریس وزن را مشابه قسمت قبل بدست می‌آوریم.

\begin{equation}
	W = 
	\left[
	\raisebox{-2.7cm}{
		\begin{tikzpicture}[node distance=1cm, every node/.style={anchor=center}]
			\def\dr{2.5}
			\def\dc{1}
			\foreach \m in {0,1,2,3,4,5} {
				\foreach \n in {0,1,2} {
					\pgfmathparse{\m + \n}%
					\pgfmathtruncatemacro{\sum}{\pgfmathresult} % Convert to integer
					\pgfmathsetmacro{\p}{(2 * (\m * sin(pi/6)*cos(pi/4) * 0.5 + \n * sin(pi/6)*sin(pi/4) * 0.8))}%
					% Calculate element and place in matrix
					\ifnum\sum=0
					\node at (\n*\dr,-\m*\dc) {$1$};
					\else
					\node at (\n*\dr,-\m*\dc) {$e^{-j (\p) \pi}$};
					\fi
				}
			}
	\end{tikzpicture}}
	\right]
\end{equation}


	
\begin{figure}[H]
	\centering
	\includegraphics[width=0.5\linewidth]{Q3/results/w2-p-plot3}
	\caption{}
	\label{fig:w2-p-plot3}
\end{figure}

\begin{figure}[H]
	\centering
	\includegraphics[width=0.5\linewidth]{Q3/results/w2-imagesc}
	\caption{}
	\label{fig:w2-imagesc}
\end{figure}

\begin{figure}[H]
	\centering
	\includegraphics[width=0.5\linewidth]{Q3/results/w2-cart}
	\caption{}
	\label{fig:w2-cart}
\end{figure}
چون طبق فرض بیشینه مقدار گین در $\phi = 24^\circ$ رخ می‌دهد برای این زاویه داریم:

\begin{figure}[H]
	\centering
	\includegraphics[width=\linewidth]{Q3/results/w2-phi-0-mag}
	\caption{}
	\label{fig:w2-phi-0-mag}
\end{figure}
\begin{figure}[H]
	\centering
	\includegraphics[width=\linewidth]{Q3/results/w2-phi-0-db}
	\caption{}
	\label{fig:w2-phi-0-db}
\end{figure}
	
	
	\begin{table}[H]
		\centering
		\begin{tabular}{ccc}
			\toprule\rowcolor{gray!10}
			\textbf{HPBW} & \textbf{NNBW} & \textbf{Gain Peak}\\\midrule
			$0.140\pi$ & $0.47\pi$ & $16 = 24 \ \text{dB}$ \\
			\bottomrule
		\end{tabular}
	\end{table}
	
	مشاهده می‌شود که یک گریتینگ لوب بزرگ پیدا می‌کند که باعث ابهام است. اما مشخصات آن زیاد تغییر نمی‌کند.
	
	
	
	
	
	
	%%%%%%%%%%%%%%%%%%%%%%%%%%%%%%%%%%%%%%%%%%%%%%%%%%%%%%%%%%%%%%%%%%%%
	\FloatBarrier\question[\lr{Phased array Radar Jamming}]%4
	%%%%%%%%%%%%%%%%%%%%%%%%%%%%%%%%%%%%%%%%%%%%%%%%%%%%%%%%%%%%%%%%%%%%
	

		
		قدرت سیگنال قابل تشخیص حداقل (\lr{\(P_{\text{min}}\)}) برای یک رادار آرایه فازی با ویژگی‌های مشخص شده در جدول زیر به صورت رابطه‌ی زیر تعریف می‌شود:
		
		\[
		P_{\text{min}} = \frac{P_t G^2 \lambda^2 \sigma}{(4 \pi)^3 R_{\text{max}}^4}
		\]
		
		که در آن:
		\begin{itemize}
			\item \(P_{\text{min}}\): حداقل قدرت سیگنال قابل تشخیص
			\item \(P_t\): قدرت سیگنال ارسالی
			\item \(G\): بهره‌ی آنتن
			\item \(\lambda\): طول موج
			\item \(\sigma\): مقطع راداری (\lr{RCS}) هدف
			\item \(R_{\text{max}}\): حداکثر برد
		\end{itemize}
		
		این معادله باید نسبت سیگنال به نویز حداقلی (\lr{SNR}) لازم برای تشخیص را تأمین کند.
		
		جدول مشخصات رادار:
		\begin{table}[h!]
			\centering
			\begin{tabular}{|c|c|c|}
				\hline
				\textbf{پارامتر} & \textbf{مقدار} & \textbf{واحد} \\
				\hline
				فرکانس (\lr{F}) & 8 & \lr{GHz} \\
				\hline
				فاکتور نویز (\lr{NF}) & 5 & \lr{dB} \\
				\hline
				برد (\lr{R}) & $1-100$ & \lr{km}\\
				\hline
				بهره المان ($G_0$) & 6 & \lr{dBi} \\
				\hline
				پهنای باند (\lr{BW}) & 10 & \lr{MHz} \\
				\hline
				حداقل \lr{SNR} (${SNR_{\text{min}}}$) & 8 & \lr{dB} \\
				\hline
				حداقل \lr{RCS} (\(\sigma_{\text{min}}\)) & $0.5$ & \(\text{m}^2\) \\
				\hline
				توان المان (${P_0}$) & 2 & \lr{W} \\
				\hline
				دمای اتاق (\(T_0\)) & 290 & \(\text{K}\) \\
				\hline
				ثابت بولتزمن (\lr{KB}) & \(1.38 \times 10^{-23}\) & \(\text{J/K}\) \\
				\hline
			\end{tabular}
			\caption{مشخصات رادار آرایه فازی}
		\end{table}

رابطه بالا در حقیقت توان سیگنال دریافتی را نشان می‌دهد. برای محاسبه توان نویز لازم است که از رابطه زیر استفاده کنیم.



\begin{equation}
	N_0 = K_B T_0 B_w F = \num{1.2655e-13} = \num{-128.9772} \ \text{dB}
\end{equation}

حال SNR کمینه به صورت زیر تعریف می‌شود:

\begin{equation}
	\text{SNR}_{\min} = \frac{P_{\min}}{N_0} 
\end{equation}


در مقیاس db رابطه فوق را می‌نویسیم.
\begin{equation}
	P_{\min} = {N_0} + \text{SNR} = \num{-128.9772} + 8 = \num{-120.9772} \ \text{dB}
\end{equation}


\begin{equation}
	P_t = N P_0 \quad,\quad G = N G_0
\end{equation}

حال از این رابطه استفاده می‌کنیم و خیلی ساده $N$ را بدست می‌آوریم.

\begin{equation}
	 N^3 = \frac{P_{\text{min}}(4 \pi)^3 R_{\text{max}}^4}{ P_0 G_0^2 \lambda^2 \sigma}
\end{equation}

حال برای بدست آوردن $N$ به طریق زیر عمل می‌کنیم.
\begin{equation}
	N = \Big\lceil\ (N^3)^{\tfrac{1}{3}}\ \Big\rceil = 19229
\end{equation}
	
	
	
	
	
	

		

		
		در این بخش، فرض می‌کنیم که یک آرایه فازی با مشخصات مشخص داریم. این آرایه شامل المان‌هایی است که به شکل متقارن در یک ساختار آرایه‌ای قرار گرفته‌اند. همچنین فرض می‌کنیم که راداری در فاصله ۵ کیلومتری از آرایه قرار گرفته که در راستای لوب فرعی آن واقع شده است. هدف ما این است که کیفیت سیگنال حفظ شود، به‌طوری‌که نسبت سیگنال به نویز (\lr{SNR$_{\text{min}}$}) در لوب فرعی بیشتر از یک مقدار مشخص باشد.
		
		برای این منظور، لازم است که مقدار \lr{SLL} (نسبت سطح لوب فرعی به سطح لوب اصلی) برای آرایه فازی محاسبه شود. این مقدار به ما کمک می‌کند تا میزان کاهش بهره در لوب فرعی را مشخص کنیم و کیفیت کلی سیگنال را ارزیابی کنیم. در این راستا، بهره لوب فرعی را به صورت \( G_l \) تعریف می‌کنیم و روابط زیر را استفاده خواهیم کرد.
		

		نسبت سیگنال به نویز برای حداقل مقدار قابل قبول از سیگنال به شکل زیر بیان می‌شود (رابطه‌ی فریس):
		\[
		SNR_{\text{min}} = P_t G_t G_r \left( \frac{\lambda}{4 \pi R} \right)^2 \frac{1}{K_B T_0 B N L}
		\]


		توجه داریم که در این حالت افت جمر از معکوس توان دوی $R$ استفاده می‌کند.
		با توجه به تعریف بهره لوب فرعی و استفاده از رابطه \( SNR_{\text{min}} \)، بهره لوب فرعی به صورت زیر محاسبه می‌شود:
		\[
		G_l = \frac{K_B T_0 B N L \, SNR_{\text{min}}}{N_t^2 P_0 G_0} \left( \frac{4 \pi R}{\lambda} \right)^2
		\]
		
		
		با استفاده از نرم‌افزار متلب و جایگذاری مقادیر، مقدار بهره لوب فرعی به صورت زیر محاسبه می‌شود:
		\[
		G_l = 7.61 \times 10^{-10}
		\]
		

		
		بهره بیشینه (\( G_{\text{max}} \)) برای آرایه فازی به صورت زیر تعریف می‌شود. این مقدار با توجه به تعداد المان‌ها و تقویت سیگنال در لوب اصلی محاسبه می‌شود:
		\[
		G_{\text{max}} = N_t \cdot G_0
		\]
		
		با توجه به اینکه آرایه ما دارای \( N_t = 20 \) المان است، مقدار بهره بیشینه برابر است با:
		\[
		G_{\text{max}} = 20 \times 3.9811 = 79.6214
		\]
		

		
		نسبت سطح لوب فرعی به سطح لوب اصلی (\( SLL \)) با استفاده از بهره‌های محاسبه‌شده به صورت زیر است:
		\[
		SLL = \frac{G_l}{G_{\text{max}}} = \num{9.5577e-12} 
		\]
		
		
		
		به‌صورت دسی‌بل، مقدار \lr{SLL} برابر خواهد بود با:
		\[
		SLL = 10 \log_{10}(SLL) = -110.1965 \, \text{dB}
		\]
		

		
		مقدار نسبت سطح لوب فرعی به سطح لوب اصلی (\( SLL \)) محاسبه شد و برابر با \( 110 \, \text{dB} \) است. این مقدار نشان‌دهنده کیفیت طراحی آرایه فازی و دقت آن در کاهش سطح لوب‌های فرعی است.
		

	
	
	
	
	
	
	
	
	
	
	
	
	
	
	
	
	
	%%%%%%%%%%%%%%%%%%%%%%%%%%%%%%%%%%%%%%%%%%%%%%%%%%%%%%%%%%%%%%%%%%%%
	\FloatBarrier\question[\lr{Array Tapering}]%5
	%%%%%%%%%%%%%%%%%%%%%%%%%%%%%%%%%%%%%%%%%%%%%%%%%%%%%%%%%%%%%%%%%%%%
	
	تابع کایزر به صورت زیر است:
	\begin{latin}
		\lstinputlisting{Q5/kaiser_window.m}
	\end{latin}
	
	که رسم شده آن به صورت زیر است:
	
	\begin{figure}[H]
		\centering
		\includegraphics[width=0.5\linewidth]{Q5/results/keiser-11}
		\caption{}
		\label{fig:keiser-11}
	\end{figure}
	
	برای پارامتر $\alpha$ آن را با آزمون خطا طوری تنظیم می‌کنم که SLL خروجی برابر با 30 db شود.
	
\begin{figure}[H]
	\centering
	\includegraphics[width=\linewidth]{Q5/results/af-with-keiser}
	\caption{}
	\label{fig:af-with-keiser}
\end{figure}


\(w[n] = [\csvreader[head=false, after line={;}]
{Q5/results/keiser.csv}{}{\num{\csvcoli}}] \) 


\begin{figure}[H]
	\centering
	\includegraphics[width=\linewidth]{Q5/results/af-db-with-keiser}
	\caption{}
	\label{fig:af-db-with-keiser}
\end{figure}
 
 مقدار $\alpha$ در شکل نمایش داده شده است. با زیادتر شدن مقدار $\alpha$ اندازه \lr{SLL} نیز افزایش پیدا می‌کند. 
 
 \begin{equation}
 	\alpha \ge 3.427
 \end{equation}
 
 
 
 در مورد تغییر خواسته شده برای سوال ۴، قسمت اول، تمامی روابط از آنجایی آمد که میخواستیم با توجه کمینه مقدار SNR مقدار توان دریافتی را پیدا کنیم. اینجا $N$ برای ایجاد توان دریافتی مناسب کاربرد داشت. و مقدار SLL روی آن نقش نداشت. پس روی قسمت الف تاثیری ندارد.
 
 اما در قسمت ب، چون SLL را می‌توانیم با پنجره کایزر بهبود بدهیم، به تعداد کمتری $N$ نیاز داریم تا 110 دیبی SLL بگیریم.


	%%%%%%%%%%%%%%%%%%%%%%%%%%%%%%%%%%%%%%%%%%%%%%%%%%%%%%%%%%%%%%%%%%%%
	\newpage
	\bibliographystyle{plainnat-fa}
	\nocite{*}
	\bibliography{references}
	
\end{document}