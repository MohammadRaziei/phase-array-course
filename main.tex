\documentclass[12pt,onecolumn,a4paper]{article}
\usepackage{amsthm,amsmath,amssymb,bm}
\usepackage{epsfig,graphicx,subcaption}
\usepackage{float}
\usepackage{color,xcolor}
\usepackage{fmtcount}
\usepackage{placeins}
\usepackage{adjustbox}
\usepackage{tikz}
\usepackage{csvsimple}
\usepackage[top=1in, left=1in, right=1in, bottom=1in]{geometry}
\usepackage{nicefrac}
\usepackage{fancyhdr}
\usepackage{listings}
\usepackage{tabularx, booktabs, makecell}
\usepackage{hyperref,url}
\usepackage{listings}
\usepackage{mathtools} % For \xlongequal
\usepackage{siunitx}
\usepackage{tcolorbox}
\usepackage[numbers]{natbib}

% Define a custom note environment
\newtcolorbox{note}{colback=lightgray!10!white, colframe=lightgray!50!black, title=Note}

\usepackage{titlesec}

\setcounter{secnumdepth}{4}

\titleformat{\paragraph}
{\normalfont\normalsize\bfseries}{\theparagraph}{1em}{}
\titlespacing*{\paragraph}
{0pt}{3.25ex plus 1ex minus .2ex}{1.5ex plus .2ex}




\makeatletter
\let\old@lstKV@SwitchCases\lstKV@SwitchCases
\def\lstKV@SwitchCases#1#2#3{}
\makeatother
\usepackage{lstlinebgrd}
\makeatletter
\let\lstKV@SwitchCases\old@lstKV@SwitchCases

\lst@Key{numbers}{none}{%
	\def\lst@PlaceNumber{\lst@linebgrd}%
	\lstKV@SwitchCases{#1}%
	{none:\\%
		left:\def\lst@PlaceNumber{\llap{\normalfont
				\lst@numberstyle{\thelstnumber}\kern\lst@numbersep}\lst@linebgrd}\\%
		right:\def\lst@PlaceNumber{\rlap{\normalfont
				\kern\linewidth \kern\lst@numbersep
				\lst@numberstyle{\thelstnumber}}\lst@linebgrd}%
	}{\PackageError{Listings}{Numbers #1 unknown}\@ehc}}
\makeatother
\newcounter{subListing}[subfigure]

\definecolor{codegreen}{rgb}{0,0.6,0}
\definecolor{codegray}{rgb}{0.5,0.5,0.5}
\definecolor{codepurple}{rgb}{0.58,0,0.82}
\definecolor{mygreen}{RGB}{28,172,0} 
\definecolor{mylilas}{RGB}{170,55,241}
\definecolor{backcolour}{rgb}{1,1,0.98}

\lstset{language=MATLAB,%
	backgroundcolor=\color{backcolour},   
	commentstyle=\color{codegreen},
	keywordstyle=\color{blue},
	numberstyle=\tiny\color{codegray},
	stringstyle=\color{codepurple},
	basicstyle=\tt\scriptsize,
	frame = LBtr,
	%frameround=T,
	rulecolor=\color{gray},
	showstringspaces=false,
	numbers=left,%
	numberstyle={\tiny\color{gray}},
	numbersep=8pt,
	breaklines=true,
	%postbreak=\mbox{\textcolor{yellow}{$\hookrightarrow$}\space},
	tabsize=2,
	escapechar=`,
	xleftmargin=1.8 em, 
	framexleftmargin=2em,
}

\newcommand*{\transpose}{{\mkern-1.5mu\mathsf{T}}}


\usepackage{titlesec}
\titleformat{\section}[block]
{\titlerule\addvspace{4pt}\normalfont\fontsize{14}{16}\bfseries}
{\thesection\enspace}{0pt}{}[\vspace{2pt}\titlerule]


\newcommand\question[1][\space]{
	\section[Question \numberstringnum{\thesection}]
	{Question \numberstringnum{\thesection}: #1}
}


\author{Mohammad Raziei}
\title{Solutions to the First Series of Exercises}
\date{\today}


\definecolor{questioncolor}{rgb}{0.1, 0.1, 0.5}


\newcounter{rownum} % Define a counter for the row number
\newcounter{csvrownum} % Define a counter for the row number

\newcommand\saverread[2]{
	%	\(
	\csvreader[head=false, 
	before reading=\setcounter{csvrownum}{1}, after line=\stepcounter{csvrownum} 
	]{#1/results/saver.csv}{}%
	{\ifnum\thecsvrownum=#2 \num{\csvcoli} \fi}
	%	\)
}



\begin{document}
	
	
	% Set the page style to "fancy"...
	\pagestyle{fancy}
	%... then configure it.
	%		\fancyhead{} % clear all header fields
	%		\fancyhead[RO,LE]{\textbf{The performance of new graduates}}
	%		\fancyfoot{} % clear all footer fields
	%		\fancyfoot[LE,RO]{\thepage}
	%		\fancyfoot[LO,CE]{From: K. Grant}
	%		\fancyfoot[CO,RE]{To: Dean A. Smith}
	\maketitle
	
	
	%%%%%%%%%%%%%%%%%%%%%%%%%%%%%%%%%%%%%%%%%%%%%%%%%%%%%%%%%%%%%%%%%%%%
	\FloatBarrier\question[Friis link equation and phased array design]%1
	%%%%%%%%%%%%%%%%%%%%%%%%%%%%%%%%%%%%%%%%%%%%%%%%%%%%%%%%%%%%%%%%%%%%
	
	\subsection{Single Antenna Implementation}\label{SISO}
	\textcolor{questioncolor}{A small-cell backhaul company wants to design a 60 GHz link between two stations which are
		$1.2 \text{km}$ apart. The objective is to transmit and receive minimum $1 \text{Gbps}$ data.}
	
	\subsubsection{Part a}
	{\color{questioncolor} If CMOS technology is used, what is the minimum antenna gain for a Single antenna at TX
	and RX (SISO)?
	\\[1em]
	\noindent Hints:
	\begin{enumerate}
		\item CMOS power amplifiers give an output power ($P_{1\text{dB}}$) of typically 0 to 8 dBm at 60 GHz. 
		\item Assume QPSK modulation, and find minimum required SNR for BER of $10^{-5}$. 
		\item Calculate the physical bandwidth assuming $25\%$ coding overhead.
		\item Find Oxygen ($O_2$) absorption loss at $60 \text{GHz}$ band.
		\item Assume clear sky condition (no rain). \textcolor{black}{$\rightarrow$ Means \(n = 2\) in Friis equation}
		\item Assume perfect antenna alignment and matching.  \textcolor{black}{$\rightarrow$ Means \(\eta_r = \eta_t = 1\) }
		\item Plot antenna gain as a function of $P_{1\text{dB}}$ and receiver noise figure (F ranges from 5 to 10 dB).
		\item You are allowed to make any reasonable assumption if necessary. But you should clearly justify your assumption.	
	\end{enumerate}
	}





\paragraph{Antenna Gain Calculation Using Friis and Shannon's Formulas}
	
	The signal-to-noise ratio (SNR) for a wireless link is expressed by the Friis link equation:
	
	\begin{align}
		\text{SNR} &= P_t \cdot G_t \cdot G_r \cdot \left( \frac{\lambda}{4 \pi \ell} \right)^n \cdot \frac{1}{K_B \cdot T_0 \cdot B_w \cdot F \cdot \mathcal{L}}
	\end{align}
	
	where:
	\begin{itemize}
		\item \(\lambda\): Carrier wavelength,
		\item \(\ell\): Link range,
		\item \(K_B = 1.38 \times 10^{-23} \, \text{J/K}\): Boltzmann constant,
		\item \(T_0 = 290 \, \text{K}\): Absolute room temperature,
		\item \(B_w\): Channel bandwidth,
		\item \(F\): Receiver noise figure,
		\item \(\mathcal{L}\): Losses in the system.
	\end{itemize}
	
	\paragraph{System Model}
	For a phased array system, the transmit power (\(P_t\)) and antenna gains (\(G_t, G_r\)) are defined as:
	
	\begin{align}
		P_t &= N_t \cdot P_{t1}, \\
		G_t &= \eta_t \cdot N_t \cdot G_{t1}, \\
		G_r &= \eta_r \cdot N_r \cdot G_{r1},
	\end{align}
	
	where:
	\begin{itemize}
		\item \(N_t\): Number of transmit antennas (\(N_t = 1\) for SISO),
		\item \(N_r\): Number of receive antennas (\(N_r = 1\) for SISO),
		\item \(\eta_t = \eta_r = 1\): Array efficiency (perfect alignment and matching).
	\end{itemize}
	
	Assuming \(G_t = G_r = G\):
	
	\begin{equation}
		G_t = G_r = G
	\end{equation}
	
	\paragraph{Oxygen Absorption Loss}
	The oxygen absorption loss is given by:
	
	\begin{equation}
		\mathcal{L} = \alpha_{\text{O}_2} \cdot d,
	\end{equation}
	
	where \(\alpha_{\text{O}_2} = 15 \, \text{dB/km}\) is the specific attenuation, and \(d = 1.2 \, \text{km}\). Substituting:
	
	\begin{equation}
		\mathcal{L} = 15 \cdot 1.2 = 18 \, \text{dB} = \saverread{Q1}{4}.
	\end{equation}
	
	\paragraph{Minimum SNR from BER Curve}
	The minimum SNR for \(\text{BER} = \num{1e-5}\) is obtained from the BER curve shown in Figure~\ref{fig:ber-snr}.
	
	\begin{figure}[H]
		\centering
		\includegraphics[width=0.5\linewidth]{Q1/results/ber-snr}
		\caption{BER vs. SNR for QPSK modulation.}
		\label{fig:ber-snr}
	\end{figure}
	
	According to Figure~\ref{fig:ber-snr}, the minimum SNR is:
	
	\begin{equation}
		\text{SNR} = \saverread{Q1}{1} \, \text{dB} = \saverread{Q1}{2}.
	\end{equation}
	
	\paragraph{Bandwidth Calculation Using Shannon's Formula}
	The channel capacity is given by:
	
	\begin{equation}
		C = B_w \log_2\left(1 + \frac{S}{N}\right),
	\end{equation}
	
	where:
	\begin{itemize}
		\item \(C = 1.25 \, \text{Gbps}\) (with 25\% coding overhead),
		\item \(\text{SNR} = \saverread{Q1}{1} \, \text{dB}\).
	\end{itemize}
	
	Rearranging for \(B_w\):
	
	\begin{equation}
		B_w = \frac{C}{\log_2\left(1 + \text{SNR}\right)} = \saverread{Q1}{3} \, \text{Hz}.
	\end{equation}
	
	\paragraph{Antenna Gain Calculation}
	Using Friis equation:
	
	\begin{equation}
		G^2 = \text{SNR} \cdot K_B \cdot T_0 \cdot B_w \cdot \mathcal{L} \cdot \frac{F}{P_t} \cdot \left( \frac{4 \pi \ell}{\lambda} \right)^2,
	\end{equation}
	
	where:
	\begin{itemize}
		\item \(\lambda = \saverread{Q1}{5} \, \text{m}\),
		\item \(P_t = 8 \, \text{dBm}\),
		\item \(F = 5 \, \text{dB}\),
		\item \(\mathcal{L} = \saverread{Q1}{4}\).
	\end{itemize}
	
	From precomputed values:
	\begin{equation}
		G^2 = \saverread{Q1}{6},
	\end{equation}
	
	\begin{equation}
		G = \sqrt{\saverread{Q1}{6}} = \saverread{Q1}{7} = \saverread{Q1}{8} \, \text{dB}.
	\end{equation}
	
	\paragraph{Conclusion}
	The minimum antenna gain for both transmit and receive antennas is approximately \(\saverread{Q1}{8} \, \text{dB}\) under the given conditions.
	


\subsubsection{Part b}
{\color{questioncolor}
	What is the receiver sensitivity?
}














\subsection{SISO:  Pole Sway and antenna misalignment}
{\color{questioncolor}


Fig. \ref{fig:lighting-poles-sway} shows pole sway caused by wind. Measurements show that pole sway can be as large as $\pm2.7^\circ$. Assume Parabolic antennas are used in Section \ref{SISO}. For a Parabolic antenna 3-dB beamwidth in degree is approximately given by:

\begin{figure}[H]
	\centering
	\includegraphics[width=0.25\linewidth]{HW/lighting-poles-sway}
	\caption{}
	\label{fig:lighting-poles-sway}
\end{figure}


\begin{equation}
	\Delta\theta=\frac{70\lambda}{L}
\end{equation}
where $\Delta\theta$ denotes antenna beamwidth and $L$ is the antenna diameter.\\
You can calculate the gain of a Parabolic antenna from here:\\ $http://www.qsl.net/pa2ohh/jsparabolic.htm$
\\ or use the following relation: 

\begin{equation}
	G=\eta\times\frac{\pi^2L^2}{\lambda^2}
\end{equation}
where $\eta$, known as the aperture efficiency, is typically 0.55 to 0.70.


How much drop in the received SNR is caused by maximum antenna pole sway? What solutions do you recommend (at least 3 solutions)?\\

Hints: \\

You should approximate the antenna gain with a linear function of angle.

}






	%%%%%%%%%%%%%%%%%%%%%%%%%%%%%%%%%%%%%%%%%%%%%%%%%%%%%%%%%%%%%%%%%%%%
\FloatBarrier\question%2
%%%%%%%%%%%%%%%%%%%%%%%%%%%%%%%%%%%%%%%%%%%%%%%%%%%%%%%%%%%%%%%%%%%%
























	
	\begin{note}
		Figure~\ref{fig:R1} and all similar figures in this question are generated automatically using the \texttt{tikz} package in \LaTeX. The procedure is as follows: the matrix $R$ in Equation~\eqref{eq:R1} is calculated in MATLAB and saved as a .csv file. Then, it is read in \LaTeX using the \texttt{csvsimple} package and plotted with the \texttt{tikz} package. This fully automates the entire process, which is worth mentioning.
	\end{note}
	
	

	
	%%%%%%%%%%%%%%%%%%%%%%%%%%%%%%%%%%%%%%%%%%%%%%%%%%%%%%%%%%%%%%%%%%%%
	\newpage
	\bibliographystyle{plainnat}
	%\nocite{*}
	\bibliography{references}
	
\end{document}