\documentclass[12pt,onecolumn,a4paper]{article}
\usepackage{epsfig,graphicx,subcaption,amsthm,amsmath}
\usepackage{float}
\usepackage{color,xcolor}
\usepackage{fmtcount}
\usepackage{placeins}
\usepackage{adjustbox}
\usepackage{tikz}
\usepackage{csvsimple}
\usepackage[top=1in, left=1in, right=1in, bottom=1in]{geometry}
\usepackage{nicefrac}
\usepackage{fancyhdr}
\usepackage{listings}

\usepackage{listings}

\makeatletter
\let\old@lstKV@SwitchCases\lstKV@SwitchCases
\def\lstKV@SwitchCases#1#2#3{}
\makeatother
\usepackage{lstlinebgrd}
\makeatletter
\let\lstKV@SwitchCases\old@lstKV@SwitchCases

\lst@Key{numbers}{none}{%
	\def\lst@PlaceNumber{\lst@linebgrd}%
	\lstKV@SwitchCases{#1}%
	{none:\\%
		left:\def\lst@PlaceNumber{\llap{\normalfont
				\lst@numberstyle{\thelstnumber}\kern\lst@numbersep}\lst@linebgrd}\\%
		right:\def\lst@PlaceNumber{\rlap{\normalfont
				\kern\linewidth \kern\lst@numbersep
				\lst@numberstyle{\thelstnumber}}\lst@linebgrd}%
	}{\PackageError{Listings}{Numbers #1 unknown}\@ehc}}
\makeatother
\newcounter{subListing}[subfigure]

\lstset{language=matlab,%
	%backgroundcolor=\color{gray!10},
	basicstyle=\tt\scriptsize,
	frame = LBtr,
	%frameround=T,
	rulecolor=\color{gray},
	showstringspaces=false,
	numbers=left,%
	numberstyle={\tiny\color{gray}},
	numbersep=8pt,
	breaklines=true,
	%postbreak=\mbox{\textcolor{yellow}{$\hookrightarrow$}\space},
	tabsize=2,
	escapechar=`,
	xleftmargin=1.8 em, 
	framexleftmargin=2em,
}



\usepackage{titlesec}
\titleformat{\section}[block]
{\titlerule\addvspace{4pt}\normalfont\fontsize{14}{16}\bfseries}
{\thesection\enspace}{0pt}{}[\vspace{2pt}\titlerule]


\newcommand\question{
	\section{Question \numberstringnum{\thesection}}
}


\author{Mohammad Raziei}
\title{Solutions to the First Series of Exercises}
\date{\today}




\newcounter{rownum} % Define a counter for the row number
\begin{document}
	
	
	% Set the page style to "fancy"...
	\pagestyle{fancy}
	%... then configure it.
	%		\fancyhead{} % clear all header fields
	%		\fancyhead[RO,LE]{\textbf{The performance of new graduates}}
	%		\fancyfoot{} % clear all footer fields
	%		\fancyfoot[LE,RO]{\thepage}
	%		\fancyfoot[LO,CE]{From: K. Grant}
	%		\fancyfoot[CO,RE]{To: Dean A. Smith}
	\maketitle
	
	
	%%%%%%%%%%%%%%%%%%%%%%%%%%%%%%%%%%%%%%%%%%%%%%%%%%%%%%%%%%%%%%%%%%%%
	\FloatBarrier\question%1
	%%%%%%%%%%%%%%%%%%%%%%%%%%%%%%%%%%%%%%%%%%%%%%%%%%%%%%%%%%%%%%%%%%%%
	
	
	\subsection{part a, b:}
	
	
	
		\begin{figure}[h]
		
		\begin{subfigure}{\linewidth}
			\centering
			\begin{adjustbox}{scale=1.2}
				\begin{tikzpicture}[scale=1, every node/.style={scale=1}]
					\def\dscale{1}
					\foreach \vertexold/\vertex in {0/0, 0/1, 1/4, 4/6} {
						\node[draw,circle,fill=black, inner sep=2pt,line width=.7] (\vertex) at (\dscale*\vertex, 0) {};
						
						% Draw a line and add the difference label if vertexold is less than vertex
						\ifnum\vertexold<\vertex
						\pgfmathsetmacro\difference{\vertex - \vertexold} % Calculate the difference
						
						% Format the output based on whether it is an integer or not
						% Format the output based on whether it is an integer or not
						\pgfmathsetmacro\differenceDisplay{int(\difference) == \difference ? int(\difference) : \difference}
						
						% Decide on what to display
						\pgfmathsetmacro\displayText{\differenceDisplay == 1 ? "" : \differenceDisplay}
						
						
						% Draw the line and display the difference
						\draw (\vertexold) -- (\vertex) node[midway, below, red] {$\displayText d$};
						\fi
					}
				\end{tikzpicture}
			\end{adjustbox}
			\caption{}
		\end{subfigure}
		
		\vspace{1.5em}
		
		\begin{subfigure}{\linewidth}
			\centering
			\begin{adjustbox}{scale=1.2}
				\begin{tikzpicture}[scale=1, every node/.style={scale=1}]
					\def\dscale{1}
					\foreach \vertexold/\vertex in {0/0, 0/1, 1/2, 2/3, 3/4, 4/5, 5/6} {
						\node[draw,circle,fill=black, inner sep=2pt,line width=.7] (\vertex) at (\dscale*\vertex, 0) {};
						
						% Draw a line and add the difference label if vertexold is less than vertex
						\ifnum\vertexold<\vertex
						\pgfmathsetmacro\difference{\vertex - \vertexold} % Calculate the difference
						
						% Format the output based on whether it is an integer or not
						% Format the output based on whether it is an integer or not
						\pgfmathsetmacro\differenceDisplay{int(\difference) == \difference ? int(\difference) : \difference}
						
						% Decide on what to display
						\pgfmathsetmacro\displayText{\differenceDisplay == 1 ? "" : \differenceDisplay}
						
						
						% Draw the line and display the difference
						\draw (\vertexold) -- (\vertex) node[midway, below, red] {$\displayText d$};
						\fi
					}
				\end{tikzpicture}
			\end{adjustbox}
			\caption{}
		\end{subfigure}
		\caption{}
	\end{figure}
	
	
	
	\begin{equation}
		\frac{d}{\lambda} = \nicefrac12
	\end{equation}
	

\begin{lstlisting}
AF_theta =@(w_n, d_lambda, theta_0) w_n * exp(1j*(0:length(w_n)-1).'* (2*pi*d_lambda * cos(theta_0)));
AF = @(w_n, d_lambda, theta) arrayfun(@(theta_0) AF_theta(w_n, d_lambda, theta_0), theta,'UniformOutput',true);
\end{lstlisting}
	
\begin{lstlisting}
w_n = [1, 1, 0, 0, 1, 0, 1];
d_lambda = 0.5;
\end{lstlisting}
	

	
	\begin{figure}[h]
		\begin{subfigure}{.48\linewidth}
			\centering
			\includegraphics[width=\linewidth]{Q1/results/AF-plot}
			\caption{}
			\label{fig:af-plot}
		\end{subfigure}
		\hfill
		\begin{subfigure}{.48\linewidth}
			\centering
			\includegraphics[width=\linewidth]{Q1/results/AF-plot-logy}
			\caption{}
			\label{fig:af-plot-logy}
		\end{subfigure}
		\caption{
			\space
		}
	\end{figure}
	
	
	\begin{figure}
		\centering
		\includegraphics[width=.6\linewidth]{Q1/results/AF-polarplot}
		\caption{}
		\label{fig:af-polarplot}
	\end{figure}
	
	
	
	\begin{figure}
		\centering
		\begin{subfigure}{\linewidth}
			\centering
			\includegraphics[width=\linewidth]{Q1/results/AF-plot-localmaxmin-a}
			\caption{}
			\label{fig:af-plot-localmaxmin-a}
		\end{subfigure}
		
		\begin{subfigure}{\linewidth}
			\centering
			\includegraphics[width=\linewidth]{Q1/results/AF-plot-localmaxmin-b}
			\caption{}
			\label{fig:af-plot-localmaxmin-b}
		\end{subfigure}
		
		\caption{
			\space
		}
	\end{figure}
	
	
	
	%%%%%%%%%%%%%%%%%%%%%%%%%%%%%%%%%%%%%%%%%%%%%%%%%%%%%%%%%%%%%%%%%%%%
	\FloatBarrier\question%2
	%%%%%%%%%%%%%%%%%%%%%%%%%%%%%%%%%%%%%%%%%%%%%%%%%%%%%%%%%%%%%%%%%%%%
	
	
	\FloatBarrier
	\subsection{part a:}
	
	
	\FloatBarrier
	\subsection{part b:}
	
	
	\begin{figure}[h]
		\centering
		\begin{subfigure}{.43\linewidth}
			\centering
			\includegraphics[width=\linewidth]{Q2/results/2d-plot-angles}
			\caption{}
			\label{fig:2d-plot-angles}
		\end{subfigure}
		\hfill
		\begin{subfigure}{.5\linewidth}
			\centering
			\includegraphics[width=\linewidth]{Q2/results/3d-plot-angles}
			\caption{}
			\label{fig:3d-plot-angles}
		\end{subfigure}
		\caption{
		}
	\end{figure}
	
	
	\begin{figure}[h]
		\centering
		\includegraphics[width=0.7\linewidth]{Q2/results/spatial-antenna-pattern}
		\caption{}
		\label{fig:spatial-antenna-pattern}
	\end{figure}
	
	\FloatBarrier
	\subsection{part c:}
	
	\begin{figure}[h]
		\centering
		\includegraphics[width=0.4\linewidth]{Q2/results/phi0-polar}
		\caption{}
		\label{fig:phi0-polar}
	\end{figure}
	
	
	\begin{figure}[h]
		\centering
		\includegraphics[width=\linewidth]{Q2/results/phi0}
		\caption{}
		\label{fig:phi0}
	\end{figure}
	
	\FloatBarrier
	\subsection{part d:}
	
	
	
	\begin{figure}[h]
		\centering
		\includegraphics[width=0.4\linewidth]{Q2/results/phi90-polar}
		\caption{}
		\label{fig:phi90-polar}
	\end{figure}
	
	
	\begin{figure}[h]
		\centering
		\includegraphics[width=\linewidth]{Q2/results/phi90}
		\caption{}
		\label{fig:phi90}
	\end{figure}
	
	
	
	\FloatBarrier
	\subsection{part e:}
	
	
	
	
	\FloatBarrier
	\subsection{part f:}
	
	
	%%%%%%%%%%%%%%%%%%%%%%%%%%%%%%%%%%%%%%%%%%%%%%%%%%%%%%%%%%%%%%%%%%%%
	\FloatBarrier
	\question%3
	%%%%%%%%%%%%%%%%%%%%%%%%%%%%%%%%%%%%%%%%%%%%%%%%%%%%%%%%%%%%%%%%%%%%
	
	
	
	\FloatBarrier
	\subsection{part a:}
	
	
	
	\FloatBarrier
	\subsection{part b:}
	
	\begin{figure}[h]
		\centering
		\begin{subfigure}{.45\linewidth}
			\centering
			\includegraphics[width=\linewidth]{Q3/results/s11}
			\caption{$S_{11}$}
			\label{fig:s11}
		\end{subfigure}
		\hfill
		\begin{subfigure}{.45\linewidth}
			\centering
			\includegraphics[width=\linewidth]{Q3/results/s12}
			\caption{$S_{12}$}
			\label{fig:s12}
		\end{subfigure}
		
		\begin{subfigure}{.45\linewidth}
			\centering
			\includegraphics[width=\linewidth]{Q3/results/s21}
			\caption{$S_{21}$}
			\label{fig:s21}
		\end{subfigure}
		\hfill
		\begin{subfigure}{.45\linewidth}
			\centering
			\includegraphics[width=\linewidth]{Q3/results/s22}
			\caption{$S_{22}$}
			\label{fig:s22}
		\end{subfigure}
		\caption{\space}
	\end{figure}	
	
	
	
	\begin{figure}
		\centering
		\includegraphics[width=.5\linewidth]{Q3/results/S-param}
		\caption{$S$}
		\label{fig:S-param}
	\end{figure}
	
	
	
	
	\FloatBarrier
	\subsection{part c:}
	
	
	\begin{figure}[h]
		\centering
		\includegraphics[width=.5\linewidth]{Q3/results/s11-10db}
		\caption{10db-Band Width}
		\label{fig:s11-10db}
	\end{figure}
	
	\FloatBarrier
	\subsection{part d:}
	
	
	%%%%%%%%%%%%%%%%%%%%%%%%%%%%%%%%%%%%%%%%%%%%%%%%%%%%%%%%%%%%%%%%%%%%
	\FloatBarrier
	\question%4
	%%%%%%%%%%%%%%%%%%%%%%%%%%%%%%%%%%%%%%%%%%%%%%%%%%%%%%%%%%%%%%%%%%%%
	
	
	\begin{equation}
		f_0 = 60 \times 10^9 Hz
	\end{equation}
	
	\begin{equation}
		\lambda_0 = \frac{c}{f_0} = \frac{3 \times 10^8}{60 \times 10^9} = 5 \times 10^{-3}\ (m) = 5\ (mm)
	\end{equation}
	
	
	\FloatBarrier
	\subsection{part a:}
	
	\begin{equation}
		R_1 = (2.5 \times 10^{-3}) 
		\begin{bmatrix}
			3 & 0 & 0 \\
			2 & 0 & 0 \\
			1 & 0 & 0 \\
			0 & 0 & 0 \\
			-1 & 0 & 0 \\
			-2 & 0 & 0 \\
			-3 & 0 & 0
		\end{bmatrix}
		=
		\begin{bmatrix}
			% Read the CSV file and insert the values into the matrix
			\csvreader[head=false, late after line=\\]{Q4/results/R1.csv}{}%
			{\csvcoli & \csvcolii & \csvcoliii}
		\end{bmatrix}
	\end{equation}
	
	\begin{equation}
		d = 2.5 \times 10^{-3} = \frac{\lambda_0}2
	\end{equation}
	
	
	\begin{figure}[h]
		\centering
		\begin{adjustbox}{scale=1.2}
			\begin{tikzpicture}[scale=1, every node/.style={scale=1}]
				\def\dscale{300} % Set the scaling factor to 300
				\setcounter{rownum}{1} % Initialize the counter
				\coordinate (o) (0,0);		
				\csvreader[head=false]{Q4/results/R1.csv}{}%
				{%
					% Draw the node at the scaled position
					\node[draw, circle, fill=black, inner sep=2pt, line width=.7] 
					(\arabic{rownum}) at (\csvcoli*\dscale, \csvcolii*\dscale) {};
					% Add the row number above each node
					\node[above] at (\arabic{rownum}) {\arabic{rownum}};
					\draw (o) -- (\arabic{rownum});
					\stepcounter{rownum} % Increment the counter after each row
				}%
			\end{tikzpicture}
		\end{adjustbox}
		\caption{\space}
	\end{figure}
	
	\begin{figure}[h]
		\centering
		\includegraphics[width=0.5\linewidth]{Q4/results/array-beam-polar-R1}
		\caption{}
		\label{fig:array-beam-polar-r1}
	\end{figure}
	
	\begin{figure}[h]
		\centering
		\includegraphics[width=\linewidth]{Q4/results/array-beam-R1}
		\caption{}
		\label{fig:array-beam-r1}
	\end{figure}
	
	
	\FloatBarrier
	\subsection{part b:}
	
	\begin{equation}
		R_2 = (5 \times 10^{-3}) 
		\begin{bmatrix}
			3 & 0 & 0 \\
			2 & 0 & 0 \\
			1 & 0 & 0 \\
			0 & 0 & 0 \\
			-1 & 0 & 0 \\
			-2 & 0 & 0 \\
			-3 & 0 & 0
		\end{bmatrix}
		=
		\begin{bmatrix}
			% Read the CSV file and insert the values into the matrix
			\csvreader[head=false, late after line=\\]{Q4/results/R2.csv}{}%
			{\csvcoli & \csvcolii & \csvcoliii}
		\end{bmatrix}
	\end{equation}
	
	
	
	\begin{equation}
		d = 5 \times 10^{-3} = \lambda_0
	\end{equation}
	
%	\begin{adjustbox}{scale=1.2}
%		\begin{tikzpicture}[scale=1, every node/.style={scale=1}]
%			\def\dscale{1}
%			\csvreader[head=false]{Q4/results/R2.csv}{}%
%			{%
%				\node[draw,circle,fill=black, inner sep=2pt,line width=.7] (????) at (\csvcoli*\dscale, \csvcolii*\dscale) {};%
%			}%
%			
%		\end{tikzpicture}
%	\end{adjustbox}
%	
	
\begin{figure}[h]
	\centering
	\begin{adjustbox}{scale=1.2}
		\begin{tikzpicture}[scale=1, every node/.style={scale=1}]
			\def\dscale{300} % Set the scaling factor to 300
			\setcounter{rownum}{1} % Initialize the counter
			
			\coordinate (o) (0,0);		
			\csvreader[head=false]{Q4/results/R2.csv}{}%
			{%
				% Draw the node at the scaled position
				\node[draw, circle, fill=black, inner sep=2pt, line width=.7] 
				(\arabic{rownum}) at (\csvcoli*\dscale, \csvcolii*\dscale) {};
				
				% Add the row number above each node
				\node[above] at (\arabic{rownum}) {\arabic{rownum}};
				\draw (o) -- (\arabic{rownum});
				\stepcounter{rownum} % Increment the counter after each row
			}%
			\let\rownum\relax
		\end{tikzpicture}
	\end{adjustbox}
	\caption{\space}
\end{figure}
	
	
	\begin{figure}[h]
		\centering
		\includegraphics[width=0.5\linewidth]{Q4/results/array-beam-polar-R2}
		\caption{}
		\label{fig:array-beam-polar-r2}
	\end{figure}
	
	\begin{figure}[h]
		\centering
		\includegraphics[width=\linewidth]{Q4/results/array-beam-R2}
		\caption{}
		\label{fig:array-beam-r2}
	\end{figure}
	
	
	\FloatBarrier
	\subsection{part c:}
	
	
	
	\begin{equation}
		R_3 = (5 \times 10^{-3})
		\begin{bmatrix}
			3 + 0.5 \times \text{randn()} & 0 & 0 \\
			2 + 0.5 \times \text{randn()} & 0 & 0 \\
			1 + 0.5 \times \text{randn()} & 0 & 0 \\
			0 + 0.5 \times \text{randn()} & 0 & 0 \\
			-1 + 0.5 \times \text{randn()} & 0 & 0 \\
			-2 + 0.5 \times \text{randn()} & 0 & 0 \\
			-3 + 0.5 \times \text{randn()} & 0 & 0
		\end{bmatrix}
		= 
		\begin{bmatrix}
			% Read the CSV file and insert the values into the matrix
			\csvreader[head=false, late after line=\\]{Q4/results/R3.csv}{}%
			{\csvcoli & \csvcolii & \csvcoliii}
		\end{bmatrix}
	\end{equation}
	
	
		
	\begin{figure}[h]
		\centering
		\begin{adjustbox}{scale=1.2}
			\begin{tikzpicture}[scale=1, every node/.style={scale=1}]
				\def\dscale{300} % Set the scaling factor to 300
				\setcounter{rownum}{1} % Initialize the counter
				\coordinate (o) (0,0);		
				\csvreader[head=false]{Q4/results/R3.csv}{}%
				{%
					% Draw the node at the scaled position
					\node[draw, circle, fill=black, inner sep=2pt, line width=.7] 
					(\arabic{rownum}) at (\csvcoli*\dscale, \csvcolii*\dscale) {};
					% Add the row number above each node
					\node[above] at (\arabic{rownum}) {\arabic{rownum}};
					\draw (o) -- (\arabic{rownum});
					\stepcounter{rownum} % Increment the counter after each row
				}%
			\end{tikzpicture}
		\end{adjustbox}
		\caption{\space}
	\end{figure}
	
	
	\begin{figure}[h]
		\centering
		\includegraphics[width=0.5\linewidth]{Q4/results/array-beam-polar-R3}
		\caption{}
		\label{fig:array-beam-polar-r3}
	\end{figure}
	
	\begin{figure}[h]
		\centering
		\includegraphics[width=\linewidth]{Q4/results/array-beam-R3}
		\caption{}
		\label{fig:array-beam-r3}
	\end{figure}
	
	
	
	\FloatBarrier
	\subsection{part d:}
	?????
	
	\begin{equation}
		R_3 = (5 \times 10^{-3})
		\begin{bmatrix}
			3 + 0.5 \times \text{randn()} & 0 & 0 \\
			2 + 0.5 \times \text{randn()} & 0 & 0 \\
			1 + 0.5 \times \text{randn()} & 0 & 0 \\
			0 + 0.5 \times \text{randn()} & 0 & 0 \\
			-1 + 0.5 \times \text{randn()} & 0 & 0 \\
			-2 + 0.5 \times \text{randn()} & 0 & 0 \\
			-3 + 0.5 \times \text{randn()} & 0 & 0
		\end{bmatrix}
		= 
		\begin{bmatrix}
			% Read the CSV file and insert the values into the matrix
			\csvreader[head=false, late after line=\\]{Q4/results/R3-2.csv}{}%
			{\csvcoli & \csvcolii & \csvcoliii}
		\end{bmatrix}
	\end{equation}
	
	
	
		\begin{figure}[h]
		\centering
		\begin{adjustbox}{scale=1.2}
			\begin{tikzpicture}[scale=1, every node/.style={scale=1}]
				\def\dscale{300} % Set the scaling factor to 300
				\setcounter{rownum}{1} % Initialize the counter
				\coordinate (o) (0,0);		
				\csvreader[head=false]{Q4/results/R3-2.csv}{}%
				{%
					% Draw the node at the scaled position
					\node[draw, circle, fill=black, inner sep=2pt, line width=.7] 
					(\arabic{rownum}) at (\csvcoli*\dscale, \csvcolii*\dscale) {};
					% Add the row number above each node
					\node[above] at (\arabic{rownum}) {\arabic{rownum}};
					\draw (o) -- (\arabic{rownum});
					\stepcounter{rownum} % Increment the counter after each row
				}%
			\end{tikzpicture}
		\end{adjustbox}
		\caption{\space}
	\end{figure}
	
	\begin{figure}[h]
		\centering
		\includegraphics[width=0.5\linewidth]{Q4/results/array-beam-polar-R3-2}
		\caption{}
		\label{fig:array-beam-polar-r3-2}
	\end{figure}
	
	\begin{figure}[h]
		\centering
		\includegraphics[width=\linewidth]{Q4/results/array-beam-R3-2}
		\caption{}
		\label{fig:array-beam-r3-2}
	\end{figure}
	
	
	
	
	
	
	
	
	%%%%%%%%%%%%%%%%%%%%%%%%%%%%%%%%%%%%%%%%%%%%%%%%%%%%%%%%%%%%%%%%%%%%
	\FloatBarrier
	\question%5
	%%%%%%%%%%%%%%%%%%%%%%%%%%%%%%%%%%%%%%%%%%%%%%%%%%%%%%%%%%%%%%%%%%%%
	
	
	
	
	
	%%%%%%%%%%%%%%%%%%%%%%%%%%%%%%%%%%%%%%%%%%%%%%%%%%%%%%%%%%%%%%%%%%%%
	\FloatBarrier
	\question%6
	%%%%%%%%%%%%%%%%%%%%%%%%%%%%%%%%%%%%%%%%%%%%%%%%%%%%%%%%%%%%%%%%%%%%
	
	
	
	
	%%%%%%%%%%%%%%%%%%%%%%%%%%%%%%%%%%%%%%%%%%%%%%%%%%%%%%%%%%%%%%%%%%%%
	\newpage
	\bibliographystyle{plainnat}
	%\nocite{*}
	\bibliography{references}
	
\end{document}